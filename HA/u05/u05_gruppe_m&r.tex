\documentclass[a4paper,10pt]{article}
\usepackage[utf8]{inputenc}
\usepackage[T1]{fontenc}
\usepackage{amsmath,amsfonts,amssymb,amscd,amsthm,xspace}
\usepackage[ngerman]{babel}
\usepackage{listingsutf8}
\usepackage{color}
\usepackage{geometry}
\usepackage{graphicx}
\usepackage{multicol}
\usepackage{pst-tree}

\geometry{a4paper, left=2cm,right=2cm,top=2cm,bottom=2cm}

\newcommand{\Authors}{Martin Lenders (Mi. 14-16), Ralf M\"uller-Zimmermann (Di. 14-16)}
\title{H\"ohere Algorithmik - 5. \"Ubungsblatt}
\author{\Authors}
\date{\today}

\newcommand{\changefont}[3]{\fontfamily{#1} \fontseries{#2} \fontshape{#3} \selectfont}

\renewcommand{\thesection}{Aufgabe \arabic{section}:}
\renewcommand{\labelenumi}{(\theenumi)}
\renewcommand{\theenumi}{\alph{enumi}}
\renewcommand{\labelenumii}{(\theenumii)}
\renewcommand{\theenumii}{\roman{enumii}}

\definecolor{lgray}{gray}{0.95}
\definecolor{purple}{rgb}{0.498,0,0.3333}
\definecolor{identifier}{rgb}{0,0,0.1}
\definecolor{string}{rgb}{0.192,0,1}
\definecolor{comment}{rgb}{0.25,0.5,0.37}

\pagestyle{myheadings}
\oddsidemargin\oddsidemargin
\markright{\Authors}

\lstset{
	tabsize=4, 
	basicstyle=\footnotesize\fontfamily{pcr}\fontseries{m}\fontshape{n}\selectfont,
	breaklines=true,
	numbers=left,
	emphstyle=\textit, 
	language=Java,
	keywordstyle=\color{purple}\textbf, 
	identifierstyle=\color{identifier},
	stringstyle=\color{string},
	showstringspaces=false,
    escapeinside={((*}{*))},
	commentstyle=\color{comment},
	extendedchars=true,
	inputencoding=utf8/latin1
}
\psset{nodesep=2pt,levelsep=2em,treesep=2em}

\begin{document}

\maketitle

\section{Varianten der Vorlesungsbeispiele}
\begin{enumerate}
	\item \begin{description}
		\item[Rekursionsanker:] Wir verwenden die Rekursionsanker des einfachen Einkaufsproblem aus der Vorlesung, da diese auch in dem erweiterten Fall Anwendung finden.\begin{align*}
			E[n, 0] = 0\\
			E[0, B] = 0
		\end{align*}
		\item[Rekursionsgleichung:] Bei diesem Einkaufsproblem, kann man sich entweder gegen einen Artikel entscheiden oder dafür. Entscheidet man sich für einen Artikel, hat man danach wieder die Wahl, ihn zu kaufen.\begin{align*}
			E[n, B] = \max\left\{E[n, B-p_n] + w_n, E[n-1, B]\right\}
		\end{align*}
		\item[Laufzeit:] Stellt man die Matrix auf, so errechnet sich jedes Element aus dem Element darüber und daneben. Daraus ergibt sich eine gesamte Laufzeit von $O(nB)$.
	\end{description}
	\item \begin{description}
		\item[Beschreibung:]
		\item[Pseudocode:]
	\end{description}
\end{enumerate}

\section{Münzwechseln}
\begin{enumerate}
	\item \begin{description}
		\item[Überlegungen]: Wenn man nur einen Münzwert zur Verfügung hat, gibt es für jeden Centwert nur eine Möglichkeit. Null Cent kann man mit beliebig vielen Münzwerten keinmal erzeugen. Es wurde dennoch $1$ als Wert geählt, da dieser die eine Möglichkeit wiederspiegelt, dass ein Münzwert ohne Rest mit den Münzen des höchsten Münzwertes gewechselt wurde. Für einen beliebigen Centwert und einer beliebigen Anzahl von Münzwerten hängt die Anzahl der möglichen Münzwechsel davon ab, wie oft der höchste Münzwert in dem Centwert passt. Dabei entsteht immer ein Rest (inklusive $0$), der ebenfalls auf verschiedene Weisen gewechselt werden kann. Wie oft der höchste Münzwert angewandt wurde, ist an der Größe des Rests erkennbar. Somit reicht es aus, die Anzahl der Wechselmöglichkeiten des Rest mit dem nächstniedrigerem Münzwert als Maximalwert über alle Reste aufzusummieren.
		\item[Rekursionsanker:] \begin{align*}
			C[0,i] = 1\\
			C[n,1] = 1
			\end{align*}
		\item[Rekursionsgleichung:] \begin{align*}
			C[n,i] = \begin{cases}
				\sum\limits_{j=0}^{\left\lfloor\frac{n}{w_i}\right\rfloor}\left( C[n-j\cdot w_i, i-1]\right), & n >=w_i \\
				C[n,i-1], & \text{sonst}
			\end{cases}
			\end{align*}
		\end{description}
	\item
\end{enumerate}

\section{Versteckte Markov-Modelle}
\begin{enumerate}
	\item
	\item
	\item
\end{enumerate}

\end{document}