\documentclass[a4paper,10pt]{article}
\usepackage[utf8x]{inputenc}
\usepackage[T1]{fontenc}
\usepackage{amsmath,amsfonts,amssymb,amscd,amsthm,xspace}
\usepackage[ngerman]{babel}
\usepackage{listingsutf8}
\usepackage{color}
\usepackage{geometry}
\usepackage{graphicx}
\usepackage{multicol}
\usepackage{pst-tree}
\usepackage{algorithmic}

\geometry{a4paper, left=2cm,right=2cm,top=2cm,bottom=2cm}

\newcommand{\Authors}{Martin Lenders (Mi. 14-16), Ralf M\"uller-Zimmermann (Di. 14-16)}
\title{H\"ohere Algorithmik - 4. \"Ubungsblatt}
\author{\Authors}
\date{\today}

\newcommand{\changefont}[3]{\fontfamily{#1} \fontseries{#2} \fontshape{#3} \selectfont}

\renewcommand{\thesection}{Aufgabe \arabic{section}:}
\renewcommand{\labelenumi}{(\theenumi)}
\renewcommand{\theenumi}{\alph{enumi}}
\renewcommand{\labelenumii}{(\theenumii)}
\renewcommand{\theenumii}{\roman{enumii}}

\definecolor{lgray}{gray}{0.95}
\definecolor{purple}{rgb}{0.498,0,0.3333}
\definecolor{identifier}{rgb}{0,0,0.1}
\definecolor{string}{rgb}{0.192,0,1}
\definecolor{comment}{rgb}{0.25,0.5,0.37}

\pagestyle{myheadings}
\oddsidemargin\oddsidemargin
\markright{\Authors}

\lstset{
	tabsize=4, 
	basicstyle=\footnotesize\fontfamily{pcr}\fontseries{m}\fontshape{n}\selectfont,
	breaklines=true,
	numbers=left,
	emphstyle=\textit, 
	language=Java,
	keywordstyle=\color{purple}\textbf, 
	identifierstyle=\color{identifier},
	stringstyle=\color{string},
	showstringspaces=false,
    escapeinside={((*}{*))},
	commentstyle=\color{comment},
	extendedchars=true,
	inputencoding=utf8/latin1
}
\psset{nodesep=2pt,levelsep=2em,treesep=2em}

\begin{document}

\maketitle

\section{Unabhängige Mengen in Pfaden}
\begin{enumerate}
\item   Gegeben sei ein Pfad $P = (V,E)$ mit $v_1,v_2,v_3 \in V$ und $w_1 = 2, w_2 = 4, w_3 = 3$. 
        Intuitiv ist zu sehen, dass die unabhängige Menge von maximalem Gewicht $S = \{v_1, v_3\}$ mit Gewicht = 5 ist. 
        Der gegebene Algorithmus wählt jedoch $v_2$ für die Menge $S$ aus und verwirft dann $v_1$ und $v_2$.
        $S$ hat damit allerdings nur ein Gewicht von 4.
        Das Ergebnis entspricht also nicht der gesuchten Menge.
\item   Gegeben sei ein Pfad $P = (V,E)$ mit $v_1,v_2,v_3,v_4,v_5 \in V$ und $w_1 = 8, w_2 = 1, w_3 = 2, w_4 = 11, w_5 = 3$. 
        Intuitiv ist zu sehen, dass die unabhängige Menge von maximalem Gewicht $S = \{v_1, v_4\}$, mit Gewicht = 19 ist. 
        Der gegebene Algorithus wählt jedoch zunächst die beiden Mengen $S_1 = \{v_1, v_3, v_5\}$ mit Gewicht = 13 und $S_2 = \{v_2, v_4\}$ mit Gewicht = 12. 
        Zurückgegeben wird dann $S_1$, welches nicht der gesuchten Menge entspricht.
\item   Sei $W(n)$ das Gewicht der unabhängigen Menge von maximalem Gewicht für den Pfad $P = (V,E)$ mit $v_1, ..., v_n \in V$ 
        und sei dies wie folgt definiert:
        \begin{align*}
            W(1) &= w_1 \\
            W(2) &= \max\{w_1,w_2\} \\
            W(i) &= \max\{W(i-1), W(i-2)+w_i\} 
        \end{align*}
        Die eigentliche unabhängige Menge kann dann mit Backtracing ermittelt werden (jedes mal, wenn $W(i-2)+w_i$ gewählt wird, füge $v_i$ zu $S$ hinzu).
        Die Laufzeit liegt in $O(n)$, sowie auch der Speicherplatzbedarf in $O(n)$ liegt.
\end{enumerate}

\section{Vorlesungsplanung}
Vorlesungen seien verkürzt dargestellt durch $[x,y]$ $x = a(i), y = e(i)$. Zeiten werden der Einfachheit halber als natürliche Zahlen dargestellt. Eine Menge von $n$ Vorlesungen sei als Liste von Vorlesungen (mit Semikolon getrennt) dargestellt.
\begin{enumerate}
\item   Gegenbeispiel:
        \[[0,1];[1,4];[2,3];[3,4];[4,5]\]
        Das maximale $M$ der Menge wäre 
        \[M = [0,1];[2,3];[3,4];[4,5],\] 
        das Kriterium wählt aber 
        \[M = [0,1];[1,4];[4,5]\]
\item   Sei $N$ die Menge der Vorlesungen und $M \subseteq N$ die größtmögliche Menge von verträglichen Vorlesungen. Ein effizienter Algorithmus, der $M$ mit dem vorgeschlagenem Kriterium auswählt ist:
        \begin{algorithmic}
        \STATE $M \gets \emptyset$
        \WHILE{$N \neq \emptyset$}
            \STATE $m \gets$ $\arg\max \{a(i)\ |\ i \in N\}$
            \STATE $M = M \cup \{m\}$
            \STATE $N = N \setminus \{n\ |\ n \in N, e(n) > a(m)\}$
        \ENDWHILE
        \RETURN $M$
        \end{algorithmic}
        \paragraph*{Korrektheit:} Setzen wir voraus, dass wir nicht wissen, das der Algorithmus korrekt ist (und damit sein Ergebnis wirklich $M$ ist) und nennen wir dessen Ergebnis daher zu Beweiszwecken in $S$ um, so wollen wir für seine Korrektheit zeigen, dass
        \[|S| = |M|.\]
        Seien $M' = (m_1, ..., m_j)$ und $S' = (s_1, ..., s_k)$ geordnete Mengen, so dass 
        \[\{m_1,...,m_j\} = M \land \{s_1,...,s_k\} = S.\] 
        Die Ordnung sei jeweils 
        \begin{itemize} 
        \item   $\forall m_{c}, m_{c+1} \in M{:}\ a(m_{c}) > a(m_{c+1}) \land e(m_{c}) > e(m_{c+1})$ bzw.
        \item   $\forall s_{c}, s_{c+1} \in M{:}\ a(s_{c}) > a(s_{c+1}) \land e(s_{c}) > e(s_{c+1})$
        \end{itemize}
        \begin{description}
        \item[Induktionsbehauptung] $\forall r \in \{1,...,j\}{:}\ a(s_r) \geq a(m_r) (\Rightarrow |S'| \geq |M'|)$
        \item[Induktionsanfang] $r = 1{:}\ a(s_1) \geq a(m_1)$, trivial, da der Algorithmus die Vorlesung mit maximalem $a(i)$ wählt.
        \item[Induktionsschritt] $r \Rightarrow r+1{:}$
        \begin{itemize}
        \item   nach \textbf{IV} gilt: $\forall r \in \{1,...,j\}{:}\ a(s_r) \geq a(m_r)$
        \item   $a(m_r) \geq e(m_{r+1}) \Rightarrow a(s_{r+1}) \geq a(m_{r+1})$ \hfill $\square$
        \end{itemize}
        \end{description}
        \textit{$|S'| = |M'|$ (Beweis durch Widerspruch)}: Angenommen $|S'| < |M'|$, dann ist $e(s_j) < e(m_j)$ und es existiert eine weitere Vorlesung $m_{j+1} \in M'$. $N$ enthält im Algorithmus also $m_{j+1}$ nachdem bereits $m_1, ..., m_j$ ausgewählt wurden. Somit ist aber $N \neq \emptyset$ und die Bedingung zum Terminieren der While-Schleife nicht erfüllt, also $S$ nicht das Ergebnis des Algorithmus. \hfill $\square$
        \paragraph*{Laufzeit:} Im schlechtesten Fall (alle anfänglich gegebenen $n$ Vorlesungen sind verträglich) ist die Laufzeit $O(n)$, da bei jedem Schleifendurchlauf nur eine Vorlesung entfernt wird.
        Im besten Fall (keine der anfänglich gegebenen $n$ Vorlesungen sind verträglich) ist die Laufzeit $O(1)$, da nur die Vorlesung mit maximalem $a(i)$ $M$ hinzugefügt wird und alle anderen verworfen werden.
\item   Gegenbeispiel:
        \[[0,1];[1,4];[1,2];[2,3];[4,5]\]
        Das maximale $M$ der Menge wäre 
        \[M = [0,1];[1,2];[2,3];[4,5],\] 
        das Kriterium wählt aber 
        \[M = [0,1];[1,4];[4,5]\]
\item   Gegenbeispiel:
        \[[0,3];[2,4];[3,6]\]
        Das maximale $M$ der Menge wäre 
        \[M = [0,3];[3,6],\] 
        das Kriterium wählt aber 
        \[M = [2,4]\]
\item   Gegenbeispiel:
        \[[0,3];[1,2];[2,3]\]
        Das maximale $M$ der Menge wäre 
        \[M = [1,2];[2,3],\] 
        das Kriterium wählt aber 
        \[M = [0,3]\]
\end{enumerate}

\section{Autobahnfahrt}
Die Straßenkarte mit den Abständen zwischen den Tankstellen kann als Graph mit Kantengewichten gesehen werden.
Hannelore und Werner können Strecken fahren auf denen der Abstand zwischen 2 Tankstellen kleiner $n$ ist. 
Entfernen wir die Kanten mit Gewicht $> n$ also aus unserem Ausgangsgraphen, so können wir den Dijkstra-Algorithmus auf den verbleibenden Graphen anwenden.
% für Pseudocode und Laufzeitanalyse hab ich grad keine Lust :P
\end{document}
