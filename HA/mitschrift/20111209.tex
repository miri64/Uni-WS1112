\begin{itemize}
 \item $\alpha(m,n)$ taucht öftes auf: $n$ Strecken, wieviele Schnittpunkte sieht man von unten $O(n \alpha(m,n))$
 \item Bester MST-Algorithmus: $O(\underbrace{m}_{\text{\# Kosten}}, \alpha(m,\underbrace{n}_{\text{\# Knoten}}))$ Zeit
\end{itemize}

\Defi Sei $D$ Datenstruktur mit Operationen $\operatorname{Op}_1, ..., \operatorname{Op}_k$.
         wenn für \emph{jede} genügend lange Folge $X$ von Operationen gilt: 
         \begin{itemize}
         \item Laufzeit von $X$ ist $a_1 \cdot t_1 + ... + a_k t_k$, wobei $o_1 = \text{\# der Operation $\operatorname{Op}_i$ in $X$}$
         \end{itemize}
         Dann sagen wir: Die \emph{amortisierte Laufzeit} für Op$_i$ ist $t_i$

Bei UNION-FIND: Amortisierte  Laufzeit für UNION und FIND ist $O(\alpha(m,n))$. Genauer (kann man zeigen):
\begin{itemize}
 \item die amortisierte Laufzeit für UNION ist $O(1)$
 \item die amortisierte Laufzeit fpr FIND ist $O(\alpha(m,n))$
\end{itemize}

\section{Hashing}
Wörterbuchproblem:
\begin{description}
 \item $K$: Schlüsselmenge
 \item $V$: Wertemenge
\end{description}
Speichere Menge von Einträgen $(k,v) \in K \times V$. Operation:
\begin{description}
 \item $\operatorname{put}(k,v)$ Setze Eintrag für $k$ auf $v$
 \item $\operatorname{delete}(k)$ Lösche Eintrag für $k$
 \item $\operatorname{get}(k)$ Suche Eintrag für $k$
\end{description}

\subsection{Hashing mit Verkettung}
\begin{itemize}
\item Hashtabelle $T$: Array der Länge $m$.
\item Hashfunktion $h{:} K \to \{0, ..., m-1\}$
\item Speichere Eintrage für $k$ an der Stelle $h(k)$ in $T$
\item \textbf{Kollision:} Schlüssel $k \neq l \in K$ mit $h(k) = h(l)$. In der Regel unvermeidbar
\item \textbf{Kollisionsbehandlung:} Verkettung. Speichere alle Einträge, die auf die selbe Stelle hashen in einer verketteten Liste
\end{itemize}
Laufzeit?
\begin{itemize}
 \item Hängt von $h$ ab
 \item Theorie: $h$ zufällige Funktion $\rightarrow$ gute Laufzeit
 \item Aber: Wie wähle ich $h$ zufällig und speichere es?\\
     \begin{itemize}
      \item Lösung 1: heuristische Hashfunktion
      \item Lösung 2: universelles Hashing
     \end{itemize}
\end{itemize}

\subsection[Universelles Hashing]{Universelles Hashing (Carter-Wegman)}
\Defi Sei $H$ eine Menge von Funktionen $h{:} K \to \{0,...,m-1\}$. $H$ heißt \emph{universell}, wenn 
\[\forall k, l \in K, k \neq l{:} \operatorname{Pr}_{h \in H}\left[h(k) = h(l)\right] \leq \frac{1}{m}\]
\Satz Sei $H$ universell, $h \in H$ zufällig. Sei $T$ Hashtabelle, in der $n$ Schlüssel gespeichert sind. Sei $k \in K$ fest. Dann ist die erwartete Anzahl an Schlüsseln $\neq k$ bei $T[h(k)] \leq \frac{n}{m}$
\Bew Sei $S$ Menge der Schlüssel in $T$.
    \begin{description}
     \item $Y_k$: Anzahe der  Schlüssel in $ \setminus \{k\}$, die auf $h(k)$ gehasht werden.
    \end{description}
    Suchen $E[Y_k]$
    \begin{itemize}
     \item Definiere für $l \in S \setminus \{k\}$
             \[X_{k,l} = \begin{cases}
                          1, & \text{falls $h(k) = h(l)$} \\
                          0, & \text{sonst}
                         \end{cases}\]
     \item Dann ist $Y_k = \sum\limits_{l \in S \setminus \{k\}} X_{k,l}$
     \item Also:
         \begin{align*}
          E[Y_k] &= \sum\limits_{l \in S \setminus \{k\}} E[X_{k,l}] \\
                 &= \sum\limits_{l \in S \setminus \{k\}} \operatorname{Pr}[h(k) = h(l)] \\
                 &\overset{\text{(Definition von $H$)}}{\leq} \sum\limits_{l \in S \setminus \{k\}} \frac{1}{m} \leq \frac{n}{m}
         \end{align*}\hfill$\square$
    \end{itemize}
Bei universellem Hashing ist die erwartete Laufzeit pro Opeation $O\left(1+\underset{\text{Belegungsfaktor}}{\frac{n}{m}}\right)$.\\
Gibt es gute universelle Familien?
\begin{itemize}
 \item Ja. Sei $K = \underbrace{\{0, ..., p-1\}}_{\mathbb{Z}_p}$, $p$ Primzahl
 \item Für $a,b \in \{0, ..., p-1\}, a \neq 0$.
 \item Definiere $h_{a,b}(k) = ((\overbrace{a \cdot k + b}^{\text{Gerade}}) \mod p) \mod m$
 \item Setze $H := \{h_{a,b}\ |\ a,b \in \{0, ..., p-1\}, a \neq 0\}\}$
\end{itemize}
\Satz $H$ ist universell
\Bew Fixiere $k, l \in K, k \in l$. Was ist $\operatorname{Pr}[h(k) = h(l)]$?
\begin{itemize}
 \item Klar: $a \cdot k + l \neq a \cdot l + n \mod p$ gibt es keine Kollision:
     \begin{itemize}
     \item Fixiere $r,s \in \{0, ..., p-1\}, r \neq s$
     \item Behauptung: Es exisitiert genau ein $a, b$. so dass 
         \[a \cdot k + b = r \pmod{p} \qquad a \cdot l + b = s \pmod{p}.\]
         \Bew
             \begin{align*}
              a k + b &= r \pmod{p}\\
              a l + b &= s \pmod{p}\\\hline
              a (k - l) &= r - s \pmod{p} \\
\Leftrightarrow      a  &= (r-s)(k-l)^{-1} \pmod{p} 
             \end{align*}
             Kann $b$ auch ausrechnen $\Rightarrow$ $a, b$ eindeutig.
     \item Kollisionen passieren nur $\mod m$:
             \begin{itemize}
             \item Nimm $r, s \in \{0, ..., p-1\}, r \neq s$. Wie viele solcher $r, s$ gibt es mit $r = s \pmod{m}$
             \item $\underbrace{P}_{\text{Möglichkeiten für $r$}} (\underbrace{\left\lceil\frac{p}{m}\right\rceil - 1}_{\text{Möglichkeiten für $s$}}) \leq \frac{p(p-1)}{m} = \frac{|H|}{m}$
             \item Anzahl der Hashfunktionen, für die $h(k) = h(l) \leq \frac{|H|}{m}$ \\
                 $\Rightarrow$ Wahrscheinlichkeit für Kollision $\leq \frac{1}{m}$
             \end{itemize}
     \end{itemize}
\end{itemize}


