\documentclass[a4paper,10pt]{scrartcl}
\usepackage[utf8x]{inputenc}
\usepackage[T1]{fontenc}
\usepackage{amsmath,amsfonts,amssymb,amscd,amsthm,xspace}
\usepackage[ngerman]{babel}
\usepackage{listingsutf8}
\usepackage{color}
\usepackage{geometry}
\usepackage{graphicx}
\usepackage{multicol}
\usepackage{pst-tree}

\geometry{a4paper, left=2cm,right=2cm,top=2cm,bottom=2cm}

\newcommand{\Authors}{Martin Lenders, Ralf M\"uller-Zimmermann}
\title{H\"ohere Algorithmik - 1. \"Ubungsblatt}
\author{\Authors}
\date{\today}

\newcommand{\changefont}[3]{\fontfamily{#1} \fontseries{#2} \fontshape{#3} \selectfont}

\renewcommand{\thesection}{Aufgabe \arabic{section}}
\renewcommand{\theenumi}{(\alph{enumi})}
\renewcommand{\theenumii}{(\roman{enumii}}

\definecolor{lgray}{gray}{0.95}
\definecolor{purple}{rgb}{0.498,0,0.3333}
\definecolor{identifier}{rgb}{0,0,0.1}
\definecolor{string}{rgb}{0.192,0,1}
\definecolor{comment}{rgb}{0.25,0.5,0.37}

\pagestyle{myheadings}
\oddsidemargin\oddsidemargin
\markright{\Authors}

\lstset{
	tabsize=4, 
	frame=tlrb, 
	basicstyle=\footnotesize\changefont{pcr}{m}{n},
	breaklines=true,
	numbers=left,
	emphstyle=\textit, 
	language=Java,
	keywordstyle=\color{purple}\textbf, 
	identifierstyle=\color{identifier},
	stringstyle=\color{string},
	backgroundcolor=\color{lgray},
	showstringspaces=false,
	commentstyle=\color{comment},
	extendedchars=true,
	inputencoding=utf8/latin1
}
\psset{nodesep=2pt,levelsep=2em,treesep=2em}

\begin{document}

\maketitle

\section{$\boldsymbol{O}$-Notation}
\section{Sammelbilder}
\begin{enumerate}
\item \begin{description}
		\item[Zu zeigen:] $E[X] = \sum\limits_{i = 1}^{n} E[X_{i}]$
		\item[Herleitung:]	\begin{align*}
								\sum\limits_{i = 1}^{n} E\left[X_{i}\right] &= \sum\limits_{i = 1}^{n}\left(\sum\limits_{j = 1}^{\infty}j * P\left(X_{i} = j\right)\right)\\
								&=\sum\limits_{j = 1}^{\infty}\left(\sum\limits_{i = 1}^{n}j * P\left(X_{i} = j\right)\right)\\
								&=\sum\limits_{j = 1}^{\infty}j * \left(\sum\limits_{i = 1}^{n}P\left(X_{i} = j\right)\right)\\
								&= E\left[\sum\limits_{i = 1}^{n}X_{i}\right]\\
								&= E\left[X\right]
							\end{align*}
	\end{description}

\item \begin{description}
		\item[Gesucht:] $E[X_{i}]$
		\item[Herleitung:] \begin{align*}
		E\left[X_{i}\right] &= \frac{1}{p_{i}}\\
		p_{i} &= \frac{n-i}{n}\\
		\\
		\Rightarrow E\left[X_{i}\right] &= \frac{n}{n-i}	
	\end{align*}\end{description}
\item \begin{description}
		\item[Zu zeigen:] $E[X] = O(n \log n)$ 
		\item[Herleitung:] \begin{align*}
		E\left[X\right] &= \sum\limits_{i = 1}^{n}E\left[X_{i}\right]\\
		&= \sum\limits_{i = 1}^{n}\frac{n}{n - i}\\
		&= n * \left(1 + \frac{1}{2} + \frac{1}{3} + \cdots + \frac{1}{n}\right)\\
		&= n * \sum\limits_{i = 1}^{n}\frac{1}{i}\\
		&= n \log n
	\end{align*}\end{description}
\end{enumerate}

\section{Varianten Mergesort}
\begin{enumerate}
\item \begin{description}
       \item[Selection Sort] sortiert die Elemente einer Liste, indem er eine
Auswahl (\textit{Selection}), die zu Beginn des Algorithmus leer ist, auff\"ullt,
indem er die Verbleibende Liste nach ihrem kleinsten Element durchsucht und
durch eine Vertauschun in die Auswahl aufnimmt.

\textbf{Worst-Case-Szenario:} Im schlimmsten Fall ist die Liste falsch
herum sortiert, da der Algorithmus so bei jedem Durchlauf den aktuellen
Suchraum von vorne bis hinten durchsuchen muss. Im ersten Durchlauf sind das
also $n-1$ Elemente, im zweiten $n-2$ usw. Die Laufzeit f\"ur den Worst Case ist
also
\[
 \sum\limits_{i = 1}^{n-1} i = \frac{(n-1) \cdot n}{2} = \frac{1}{2} \left(n^2 - n\right) \in O(n^2).
\]


       \item[Mergesort] sortiert die Elemente einer Liste, indem er eine Liste
in seine Elemente zerlegt und rekursiv diese wieder zusammensetzt, wobei er
jedes mal beim zusammensetzen die neu entstandenen Teillisten sortiert.

\textbf{Worst-Case-Szenario:}  Im schlimmsten Fall müssen die Elemente bei jedem Verschmelzen wechselweise eingefügt werden.
Für das Verschmelzen von zwei $k$-elementigen Listen sind also $2k-2$ Vergleiche nötig, womit für einen Rekursionsschritt mit einer $n$-elementigen $n - 1$ Vergleiche durchgeführt werden.
Das Verschmelzen bei einer $n$-elementigen Folge wird $\log_2 n$ mal aufgerufen.
Die Laufzeit für den Worst Case ist also
\[
 (n-1) \cdot \log_2 n \in O(n \log n)
\]
      \end{description}
\item \begin{enumerate}
       \item Wir nehmen an, dass $n$ und $M$ Zweierpotenzen sind. Für den normalen Mergesort würden wir $\log_2 n$-mal die Listen verschmelzen (s. 3a). Davon entfallen aber nun die Rekursionsschritte auf die Teillisten, die der Selectionsort nun übernimmt, wir haben also nur $\log_2 n - \log_2 (M-1) = \log_2\left(\frac{n}{M-1}\right)$ Verschmelzungen. Der Selectionsort sortiert alle Teillisten der Länge $m = \frac{M}{2}$ und benötigt dafür $\frac{1}{2}(m^2-m)$ Vergleiche (s. 3a). Die Gesamtanzahl an Vergleichen ist daher:
             \[\log_2\left(\frac{n}{M-1}\right) + \frac{1}{2}\left[\frac{M^2}{4}-\frac{M}{2}\right] \in O(n \log n)\]
       \item Wir ziehen für den Laufzeitvergleich zusätzlich zu den Vergleichen des $M$-Mergesorts auch noch die $\frac{n}{M-1}-1$ Verschmelzungen mit ein:
             \[
             \]
             \begin{align*}
              \log_2\left(\frac{n}{M-1}\right) \cdot \left(\frac{n}{M-1}-1\right) + \frac{1}{2}\left[\frac{M^2}{4}-\frac{M}{2}\right] &< (n-1) \cdot \log_2 n \\
             \end{align*}
      \end{enumerate}
\item foobar
      \lstinputlisting[caption=Sort.java]{src/Sort.java}
      \lstinputlisting[caption=SelectionSort.java]{src/SelectionSort.java}
      \lstinputlisting[caption=MergeSort.java]{src/MergeSort.java}
      \lstinputlisting[caption=MMergeSort.java]{src/MMergeSort.java}
\end{enumerate}
\end{document}