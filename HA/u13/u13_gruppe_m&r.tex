\documentclass[a4paper,10pt]{article}
\usepackage[utf8]{inputenc}
\usepackage[T1]{fontenc}
\usepackage{amsmath,amsfonts,amssymb,amscd,amsthm,xspace}
\usepackage[ngerman]{babel}
\usepackage{geometry}
\usepackage{algorithmic}

\geometry{a4paper, left=2cm,right=2cm,top=2cm,bottom=2cm}

\newcommand{\Authors}{Martin Lenders (Di. 14-16), Ralf Müller-Zimmermann (Di. 14-16)}
\title{Höhere Algorithmik - 13. Übungsblatt}
\author{\Authors}
\date{\today}

\newcommand{\changefont}[3]{\fontfamily{#1} \fontseries{#2} \fontshape{#3} \selectfont}

\renewcommand{\thesection}{Aufgabe \arabic{section}:}
\renewcommand{\labelenumi}{(\theenumi)}
\renewcommand{\theenumi}{\alph{enumi}}
\renewcommand{\labelenumii}{(\theenumii)}
\renewcommand{\theenumii}{\roman{enumii}}

\renewcommand{\algorithmicrequire}{\textbf{Eingabe:}}
\renewcommand{\algorithmicensure}{\textbf{Ausgabe:}}
\renewcommand{\algorithmiccomment}[1]{\textit{//\ #1}}

\pagestyle{myheadings}
\oddsidemargin\oddsidemargin
\markright{\Authors}

\begin{document}

\maketitle

\section{Simplex}
Wir gehen bei beiden Aufgaben nach dem Vorgehen im Cormen \cite[S. 793ff]{cormen} vor. Der Simplex-Algorithmus sei hier kurz umrissen:
\begin{center}
\begin{minipage}{\textwidth}
    $\textsc{Simplex}(A, b, c)$
    \begin{algorithmic}[1]
        \REQUIRE $A \in \mathbb{R}^{m \times n}, b \in \mathbb{R}^{m}, c \in \mathbb{R}^{n}$
        \ENSURE $\bar x \in \mathbb{R}^{n}$, wobei $\bar x$ die Funktion $c^T \bar x$ mit $A\bar x \leq b, \bar x \geq 0$ maximiert
        \STATE $(A,b,c,N,B,v) \gets \textsc{Schlupfform}(A,b,c)$
        \WHILE{$\exists j \in N{:}\ c_j > 0$}
            \STATE wähle $e \in N$ mit $c_e > 0$
            \FOR{$i \in B$}
                \IF{$a_{ie} > 0$}
                    \STATE $\Delta_i \gets \frac{b_i}{b_{ie}}$
                \ELSE
                    \STATE $\Delta_i \gets \infty$
                \ENDIF
            \ENDFOR
            \STATE wähle $l \in N$, sodass $\Delta_l = \min\limits_i \Delta_i$
            \IF{$\Delta_l = \infty$}
                \RETURN "`unbeschränkt"'
            \ELSE
                \STATE $(A,b,c,N,B,v) \gets \textsc{Pivot}(A,b,c,N,B,v,l,e)$
            \ENDIF
        \ENDWHILE
        \FOR{$i \in \{1, ..., n\}$}
            \IF{$i \in B$}
                \STATE $\bar x_i \gets b_i$
            \ELSE
                \STATE $\bar x_i \gets 0$
            \ENDIF
        \ENDFOR
        \RETURN ($\bar x_1, ..., \bar x_n$)
    \end{algorithmic}
\end{minipage}
\end{center}
Wobei \textsc{Schlupfform} eine spezielle Schreibweise des LPs zurückgibt mit $A = (a_{n+i,j}), b = (b_{n+i}), c = (c_i), $
$N = \{1, ..., n\}, B = \{n+1, ..., m{+}n\}, N \cap B = \emptyset, v \in \mathbb{R}$ (fgwl. mit 0 initialisiert). $B$ entspricht dabei den "`Ecken"' des in der VL vorgestellten Verfahrens, wenn man die Zeilenindizes in der Matrix jeweils als Bezeichner für die Ungleichung sieht. \textsc{Pivot} ist wie folgt implementiert (s. nächste Seite):
\begin{center}
\begin{minipage}{\textwidth}
    $\textsc{Pivot}(A,b,c,N,B,v,l,e)$
    \begin{algorithmic}
        \REQUIRE $A \in \mathbb{R}^{m \times n}, b \in \mathbb{R}^{m}, c \in \mathbb{R}^{n}, N, B \subseteq \{1, ..., m{+}n\},N \cap B = \emptyset, v \in \mathbb{R}, l \in B, e \in N$
        \ENSURE transformierte Schlupfform mit $B$ als neue "`Ecke"'.
        \STATE \COMMENT{Berechne die Koeffizienten der Nebenbedingungen}
        \STATE $b'_e \gets \frac{b_l}{a_{le}}$
        \FOR{$j \in N - \{e\}$}
            \STATE $a'_{ej} \gets \frac{a_{lj}}{a_{le}}$ 
        \ENDFOR
        \STATE $a'_{el} \gets \frac{1}{a_{le}}$
        \FOR{$i \in B \setminus \{l\}$}
            \STATE $b'_i \gets b_i - a_{ie}b'_e$
            \FOR{$j \in N \setminus \{e\}$}
                \STATE $a'_{ij} \gets a_{ij} - a_{ie}a'_{ej}$
            \ENDFOR
            \STATE $a_{il} \gets -a_{ie}a'_{el}$
        \ENDFOR
        \STATE \COMMENT{Berechne die Koeffizienten der Zielfunktion}
        \STATE $v' \gets v + c_e b'_e$
        \FOR{$j \in N \setminus \{e\}$}
            \STATE $c'_j \gets c_j - c_ea'_{ej}$
        \ENDFOR
        \STATE $c'_l \gets c_e a'_{el}$
        \STATE \COMMENT{Neuberechungn der Mengen $N$ und $B$}
        \STATE $N' \gets N \setminus \{e\} \cup \{l\}$
        \STATE $B' \gets B \setminus \{l\} \cup \{e\}$
        \RETURN $(A',b',c',N',B',v')$
    \end{algorithmic}
\end{minipage}
\end{center}
Für die genaue Analyse dieses Vorgehens sei bitte \cite{cormen} zu Rate zu ziehen, da dies sonst den Rahmen dieses Übungsblattes sprengen würde.
\begin{enumerate}
\item   \begin{itemize}
        \item   Wir überführen das Programm zunächst in die Schlupfform
                \begin{align*}
                    z   &=     3x_1 + 2x_2 + 4x_3 \\
                    x_4 &= 4 -  x_1 -  x_2 -  x_3 \\
                    x_5 &= 5 - 2x_1        - 3x_3 \\
                    x_6 &= 7 - 2x_1 -  x_2 - 3x_3 \\
                      0 &\leq x_1, x_2, x_3, x_4, x_5, x_6 
                \end{align*}
                $\Rightarrow N = \{1,2,3\}, B = \{4,5,6\}, v = 0,$
                \[
                 A = \begin{pmatrix}
                        a_{41} & a_{42} & a_{43} \\
                        a_{51} & a_{52} & a_{53} \\
                        a_{61} & a_{62} & a_{63}
                     \end{pmatrix}
                   = \begin{pmatrix}
                        1 & 1 & 2 \\
                        2 & 0 & 3 \\
                        2 & 1 & 3
                     \end{pmatrix}, 
                 b = \begin{pmatrix}b_4 \\ b_5 \\ b_6 \end{pmatrix} 
                   = \begin{pmatrix}4 \\ 5 \\ 7 \end{pmatrix},
                 c = \begin{pmatrix}c_1 \\ c_2 \\ c_3 \end{pmatrix} 
                   = \begin{pmatrix}3 \\ 2 \\ 1 \end{pmatrix}
                \]
        \item   Wähle $e := 1$ aus $N$ (da $c_1 = 3 > 0$)
                \begin{align*}
                    \Delta_4 &:= \frac{b_4}{a_{41}} = 4 \\
                    \Delta_5 &:= \frac{b_5}{a_{51}} = \frac{5}{2} \\
                    \Delta_6 &:= \frac{b_6}{a_{61}} = \frac{7}{2} \\
                \end{align*}
        \item   Wähle $\Delta_l = \min\{\Delta_i\ |\ i \in B\} = \Delta_5$
        \item   Berechne transformierte Schlupfform mit $(N',B',A',b',c',v') = \textsc{Pivot}(N,B,A,b,c,v,l,e)$
                \begin{itemize}
                \item   Berechne neue Koeffizienten für $x'_1$ und die übrigen Nebenbedingungen
                \begin{align*}
                    b'_1    &= \frac{b_5}{a_{51}} = \frac{5}{2}    & a'_{12} &= \frac{a_{52}}{a_{51}} = 0 \\
                    a'_{13} &= \frac{a_{53}}{a_{51}} = \frac{3}{2} & a'_{15} &= \frac{1}{a_{51}} = \frac{1}{2} \\
                    \hline
                    b'_4    &= b_4 - a_{41}b'_1 = \frac{3}{2}      & b'_6    &= b_6 - a_{61}b'_1 = 2 \\
                    a'_{42} &= a_{42} - a_{41}a'_{12} = 1          & a'_{62} &= a_{62} - a_{61}a'_{12} = 1 \\
                    a'_{43} &= a_{43} - a_{41}a'_{13} = \frac{1}{2}& a'_{63} &= a_{63} - a_{61}a'_{13} = 0 \\
                    a'_{45} &= -a_{41}a'_{15} = -\frac{1}{2}       & a'_{65} &= -a_{61}a'_{15} = -1        \\
                \end{align*}
                \item   Berechne neue Zielfunktion
                \begin{align*}
                    v'   &= v + c_1b'_{1} = \frac{15}{2}             & c'_2 &= c_2 - c_1a'_{12} = 2 \\
                    c'_3 &= c_3 - c_1a'_{13} = -\frac{7}{2}          & c'_5 &= -c_1a'_{15} = -\frac{3}{2}
                \end{align*}
                \item   Damit sind:
                        $N' = \{2,3,5\}, B' = \{1,4,6\}, v' = \dfrac{15}{2},$
                        \[
                        A' = \begin{pmatrix}
                                a'_{12} & a'_{13} & a'_{15} \\
                                a'_{42} & a'_{43} & a'_{45} \\
                                a'_{62} & a'_{63} & a'_{65}
                            \end{pmatrix}
                        = \begin{pmatrix}
                                0 & \frac{3}{2} & \frac{1}{2} \\
                                1 & \frac{1}{2} & -\frac{1}{2} \\
                                1 & 0           & -1
                            \end{pmatrix}, 
                        b' = \begin{pmatrix}b'_1 \\ b'_4 \\ b'_6 \end{pmatrix} 
                           = \begin{pmatrix}\frac{5}{2} \\ \frac{3}{2} \\ 2 \end{pmatrix},
                        c' = \begin{pmatrix}c'_2 \\ c'_3 \\ c'_5 \end{pmatrix} 
                           = \begin{pmatrix}2 \\ -\frac{7}{2} \\ -\frac{3}{2} \end{pmatrix}
                        \]
                \end{itemize}
        \item   Setze $e := 2$, da $c'_2 = 2 > 0$
                \begin{align*}
                    \Delta_1 &:= \infty\\
                    \Delta_4 &:= \frac{b'_4}{a'_{42}} = \frac{3}{2}\\
                    \Delta_6 &:= \frac{b'_6}{a'_{62}} = 2\\
                                &\Rightarrow l := 4
                \end{align*}
                $N'' = \{3,4,5\}, B'' = \{1,2,6\}, v'' = \frac{21}{2},$
                \[
                A'' = \begin{pmatrix}
                        a''_{13} & a''_{14} & a''_{15} \\
                        a''_{23} & a''_{24} & a''_{25} \\
                        a''_{63} & a''_{64} & a''_{65}
                    \end{pmatrix}
                = \begin{pmatrix}
                        \frac{3}{2} & 0 & \frac{1}{2} \\
                        \frac{1}{2} & 1 & -\frac{1}{2} \\
                        -\frac{1}{2} & -1 & -\frac{1}{2}
                    \end{pmatrix}, 
                b'' = \begin{pmatrix}b''_1 \\ b''_2 \\ b''_6 \end{pmatrix} 
                    = \begin{pmatrix}\frac{5}{2} \\ \frac{3}{2} \\ \frac{1}{2} \end{pmatrix},
                c'' = \begin{pmatrix}c''_3 \\ c''_4 \\ c''_5 \end{pmatrix} 
                    = \begin{pmatrix}-\frac{9}{2} \\ -2 \\ -\frac{1}{2}  \end{pmatrix}
                \]
        \item   Simplex terminiert, da kein $j \in N''$ mit $c_j > 0$\\
                $\Rightarrow$ Ergebnisvektor $\bar x = \begin{pmatrix}b''_1 \\ b''_2 \\ 0\end{pmatrix} = \begin{pmatrix}\frac{5}{2} \\ \frac{3}{2} \\ 0\end{pmatrix}$
        \end{itemize}
\item   \begin{itemize}
        \item   Wir überführen das Programm zunächst wieder in die Schlupfform
                \begin{align*}
                    z   &=     5x_1 + 6x_2 + 9x_3 + 8x_4 \\
                    x_5 &= 5 -  x_1 - 2x_2 - 3x_3 -  x_4 \\
                    x_6 &= 3 -  x_1 -  x_2 - 2x_3 - 3x_4 \\
                      0 &\leq x_1, x_2, x_3, x_4, x_5, x_6 
                \end{align*}
                $\Rightarrow N = \{1,2,3,4\}, B = \{5,6\}, v = 0,$
                \[
                 A = \begin{pmatrix}
                        a_{51} & a_{52} & a_{53} & a_{54}\\
                        a_{61} & a_{62} & a_{63} & a_{64}
                     \end{pmatrix}
                   = \begin{pmatrix}
                        1 & 2 & 3 & 1 \\
                        1 & 1 & 2 & 3
                     \end{pmatrix}, 
                 b = \begin{pmatrix}b_5 \\ b_6 \end{pmatrix} 
                   = \begin{pmatrix}5 \\ 3 \end{pmatrix},
                 c = \begin{pmatrix}c_1 \\ c_2 \\ c_3 \\ c_4\end{pmatrix} 
                   = \begin{pmatrix}5 \\ 6 \\ 9 \\ 8 \end{pmatrix}
                \]
        \item   Setze $e := 1$, da $c_1 = 3 > 0$
                \begin{align*}
                    \Delta_5 &:= 5\\
                    \Delta_6 &:= 3 & \Rightarrow l := 6
                \end{align*}
                $\Rightarrow N' = \{2,3,4,6\}, B' = \{1,5\}, v' = 15,$
                \[
                 A' = \begin{pmatrix}
                        a'_{12} & a'_{13} & a'_{14} & a'_{16}\\
                        a'_{52} & a'_{53} & a'_{54} & a'_{56}
                     \end{pmatrix}
                   = \begin{pmatrix}
                        1 & 2 &  3 &  1 \\
                        1 & 1 & -2 & -1
                     \end{pmatrix}, 
                 b' = \begin{pmatrix}b'_1 \\ b'_5 \end{pmatrix} 
                   = \begin{pmatrix}3 \\ 2\end{pmatrix},
                 c' = \begin{pmatrix}c'_2 \\ c'_3 \\ c'_4 \\ c'_6\end{pmatrix} 
                   = \begin{pmatrix}1 \\ -1 \\ -7 \\ -5 \end{pmatrix}
                \]
        \item   Setze $e := 2$, da $c'_2 = 1 > 0$
                \begin{align*}
                    \Delta_1 &:= 3\\
                    \Delta_5 &:= 2 & \Rightarrow l := 5
                \end{align*}
                $\Rightarrow N'' = \{3,4,5,6\}, B'' = \{1,2\}, v'' = 17,$
                \[
                 A'' = \begin{pmatrix}
                        a''_{13} & a''_{14} & a''_{15} & a''_{16}\\
                        a''_{23} & a''_{24} & a''_{25} & a''_{26}
                     \end{pmatrix}
                   = \begin{pmatrix}
                        1 &  5 & -1 &  2 \\
                        1 & -2 &  1 & -1
                     \end{pmatrix}, 
                 b'' = \begin{pmatrix}b''_1 \\ b''_2 \end{pmatrix} 
                   = \begin{pmatrix}1 \\ 2\end{pmatrix},
                 c'' = \begin{pmatrix}c''_3 \\ c''_4 \\ c''_5 \\ c''_6\end{pmatrix} 
                   = \begin{pmatrix}-2 \\ -5 \\ -1 \\ -3 \end{pmatrix}
                \]
        \item   Simplex terminiert, da kein $j \in N''$ mit $c_j > 0$\\
                $\Rightarrow$ Ergebnisvektor $\bar x = \begin{pmatrix}b''_1 \\ b''_2 \\ 0 \\ 0\end{pmatrix} = \begin{pmatrix}1 \\ 2 \\ 0 \\ 0\end{pmatrix}$
        \end{itemize}
\end{enumerate}

\bibliographystyle{abbrv}
\bibliography{u13_gruppe_m&r}


\section{Lösbarkeit von LPen}


\section{Modellieren mit LPen}

\end{document}