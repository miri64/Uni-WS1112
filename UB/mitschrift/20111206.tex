\newcommand{\FIRST}{\operatorname{FIRST}}
\newcommand{\FOLLOW}{\operatorname{FOLLOW}}

\Satz Wenn Parsetabelle $M$ zu kontextfreier Grammatik $G$ höchstens \emph{einen} Eintrag in jedem Feld besitzt, dann ist $G$ eindeutig und eine LL(1)-Grammatik.
\Bsp Dangling \textbf{else}:
\begin{align*}
 S &\to iEtSS'\ |\ a \\
 S' &\to eS\ |\ \varepsilon \\
 E &\to b
\end{align*}
\begin{itemize}
 \item \textbf{Eingabe:} $ib\underbrace{\underbrace{ibta}ea}$
\end{itemize}
\begin{align*}
 \FOLLOW(S') &= \{\$, e\}
\end{align*}

\begin{center}
    \begin{tabular}{r|c|c|c|c|c|c|}
     $\mathcal{M}$ & $i$            & $t$ & $e$            & $a$       & $b$       & $\$$                 \\\hline
     $S$           & $S \to iEtSS'$ &     &                & $S \to a$ &           &                      \\\hline
     $S'$          &                &     & \begin{minipage}{2cm}
                                             $S' \to eS$ \\
                                             $S' \to \varepsilon$
                                            \end{minipage} &           &           & $S' \to \varepsilon$ \\\hline
     $E$           &                &     &                &           & $E \to b$ &                      \\\hline
    \end{tabular}
\end{center}
\begin{itemize}
 \item \textbf{Methode brutale:} Lösche aus dem Feld mit zwei Einträgen, die Regel $S' \to \varepsilon$
\end{itemize}

\paragraph*{Algorithmus} (Tabellengesteuerte, prädiktibve Syntaxanalyse)
\begin{algorithmic}
 \REQUIRE $w \in \mathcal{T}^*$ gefolgt von \$ mit $\$ \in \mathcal{T}$ und Parsetabelle $\mathcal{M}$ zu gegegebener Grammatik $G$
 \ENSURE  Wenn $w \in \mathcal{L}(G)$ Linksableitung von $w$, Fehler sonst.
 \STATE Anfangszustand: Eingabe enthält $w\$$, Stack enthält $S\$$ ($S$ über $\$$)
 \STATE $a \gets$ first symbol of $w$
 \STATE $X \gets$ top of the stack
 \WHILE{$X \neq \$$}
  \IF{$X = a$}
   \STATE pop den Stack
   \STATE $a \gets$ nächstes Eingabesymbol
  \ELSIF{$X$ ist Terminal}
   \STATE Fehlermeldung
  \ELSIF{$\mathcal{M}[X,a]$ ist leer}
   \STATE Fehlermeldung
  \ELSIF{$\mathcal{M}[X,a]$ enthält $X \to Y_1...Y_k$}
   \STATE pop Stack
   \STATE push $Y_k, ..., Y_1$ auf Stack
   \STATE gib $X \to Y_1...Y_k$ aus
  \ENDIF
  \STATE Setze $X$ auf top of stack
 \ENDWHILE
 \IF{$a = \$$}
  \STATE accept
 \ENDIF
\end{algorithmic}

\begin{center}
 \begin{tabular}{l|l|l}
  \textbf{Eingabe} & \textbf{Stack}        & \textbf{Ausgabe} \\\hline
  $ibtibtaea\$$    & $S\$$                 & $S \to iEtSS'$   \\
                   & $iEtSS'\$$            &                  \\
  $btibtaea\$$     & $EtSS'\$$             & $E \to b$        \\
                   & $btSS'\$$             &                  \\
  $tibtaea\$$      & $tSS'\$$              &                  \\
  $ibtaea\$$       & $SS'\$$               & $S \to iEtSS'$   \\
                   & $\cancel{i}EtSS'S'\$$ & $E \to b$        \\
                   & $\cancel{b}tSS'S'\$$  & $S \to a$        \\
                   & $\cancel{a}S'S'\$$    & $S' \to eS$      \\
                   & $\cancel{e}SS'\$$     & $S \to a$        \\
                   & $\cancel{a}S'\$$      & $S' \to \varepsilon$ \\
                   & $\$$                  & accept!
 \end{tabular}
\end{center}

\subsection{Fehlerbehandlung (panic mode) (Heuristiken)}
\begin{enumerate}
 \item Nimm alle Symbole aus $\FOLLOW(A)$ zu Synchronisationsmenge für $A$. \\
       Überspringe im Fehlerfall alle Eingabesymbole bis zum nächsten Synchronisationssymbol. \\
       Entferne $A$ vom Stack.
 \item (Reflektiere hierarchische Sprachstruktur)
 \item Füge $\FIRST(A)$ zur Synchronisationsmenge von $A$. Überspringe alle Eingabesymbole bist zu einem Terminal aus $\FIRST(A)$, entferne $A$ \emph{nicht} vom Stack!
 \item (Wende $\varepsilon$-Produktionen im Default-Fall an)
 \item Wenn Kellerspitze $b$ ein Terminal ungleich aktueller Eingabe ist, dann entferne $b$ vom Keller und gib Warnung aus: "`$b$ wurde eingefügt!"'
\end{enumerate}

Parsetabelle für $G$ (auf arithmetische Ausdrücke) mit Synchronisationseinträgen
\begin{align*}
 \FIRST(E) = \FIRST(T) = \FIRST(F) &= \{(,\textbf{id}\} \\
 \FIRST(E') &= \{+,\varepsilon\} \\
 \FIRST(T') &= \{*,\varepsilon\} \\\hline
 \FOLLOW(E) = \FOLLOW(E') &= \{),\$\} \\
 \FOLLOW(T) = \FOLLOW(T') &= \{+,),\$\} \\
 \FOLLOW(F) &= \{+, *, ), \$\}
\end{align*}

\begin{center}
    \newcommand{\synch}{\underline{\textbf{synch}}}
    \begin{tabular}{r|c|c|c|c|c|c|}
             & +                    & *             & (           & )                    & \textbf{id} & \$                   \\\hline
        $E$  &                      &               & $E \to TE'$ & \synch               & $E \to TE'$ & \synch               \\\hline
        $E'$ & $E' \to +TE'$        &               &             & $E' \to \varepsilon$ &             & $E' \to \varepsilon$ \\\hline
        $T$  & \synch               &               & $T \to FT'$ & \synch               & $T \to ET'$ & \synch               \\\hline
        $T'$ & $T' \to \varepsilon$ & $T' \to *FT'$ &             & $T' \to \varepsilon$ &             & $T' \to \varepsilon$ \\\hline
        $F$  & \synch               & \synch        & $F \to (E)$ & \synch               & $F \to \textbf{id}$ & \synch       \\\hline
    \end{tabular}
\end{center}
Wende Heuristiken 1, 3 und 5 an. Genügt im Allgemeinen für Ausdrücke.
\begin{center}
    \begin{tabular}{c|c|c}
        Stack & Eingabe & Ausgabe \\\hline
    \end{tabular}
\end{center}




