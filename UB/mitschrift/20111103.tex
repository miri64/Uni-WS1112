\begin{center}
\begin{pspicture}(-3,-1)(3,3)
 \rput(0,3){$S$}
 \psline[linestyle=dotted](0,2.7)(0,2.3)
 \psline[linestyle=dotted](-0.3,2.7)(-3,0.5)
 \psline[linestyle=dotted](0.3,2.7)(3,0.5)
 \rput(0,2){$A$}
  \psline[linestyle=dotted](-0.3,1.7)(-1,0.5)
  \psline[linestyle=dotted](0.3,1.7)(1,0.5)
 \rput[r](-3.1,0.25){Quellprogramm} \psframe(-3,0)(3,0.5)
 \psbrace[rot=90](-3,0)(-1,0){Terminal}
  \psline(-1,0)(-1,0.5)\rput(-0.75,0.25){$\alpha_2$}
  \psline(-2.5,0)(-2.5,0.5)
  \psline(-2,0)(-2,0.5)
  \psline(-1.5,0)(-1.5,0.5)
  \psline(-0.5,0)(-0.5,0.5)
  \psline(1,0)(1,0.5)\rput(0.25,0.25){$...$}
    \psline{->}(-0.25,-0.7)(-0.75,0)
    \psline{->}(-0.25,-0.7)(-0.25,0)
    \rput(-0.25,-1){Lookahead}
        \psline[linestyle=dashed]{->}(0.7,-1)(2,-1)
        \end{pspicture}
        
        \end{center}
\begin{align*}
 A &\to \alpha\ |\ \beta\ |\ \gamma \\
 \text{einfach } A &\to \alpha_1\alpha_2\ |\ \underline{\underline{\alpha_2\alpha_2}}\ |\ \alpha_3\alpha_3 \quad\text{mit }\alpha_1 \neq \alpha_2, \alpha_1 \neq \alpha_3
\end{align*}
\begin{description}
\item[Top-Down] (von links nach rechts)
\item[Verfahren]
    \begin{itemize}
     \item Schreibe rekursive Prozedur, zu jedem Nichtterminal genau eine
     \item Wähle im Prozedurrumpf die geeignete Regel anhand des Lookahead aus
     \item Transformiere entsprechende rechte Regelseite in Folge von match-Aufrufen bzw. rekursiven Aufrufen.
    \end{itemize}
\item[Typisches Problem] Linksrekursion
    \begin{description}
     \item[Beispiel] $E \to E+0\ |\ E+1\ |\ 0\ |\ 1$
     \item[Einsicht] $\texttt{FIRST}(\alpha)$ und $\texttt{FIRST}(\beta)$ soll disjunkt sein, wenn es zwei Regeln der Form $A \to \alpha\ |\ \beta$ gibt
     \item[Definition] $\texttt{FIRST}(\alpha) = \{a \in \mathcal{T}\ |\ \alpha \overset{*}{\mapsto} a\alpha'\} \cup \{\varepsilon\ |\ \alpha \overset{*}{\mapsto} \varepsilon\}$ \psframe(0,0)(1.5,0.5)\psline(0.5,0)(0.5,0.5)\psline(1,0)(1,0.5)\psline(1.5,0)(1.5,0.5)\rput(0.25,0.25){1}\rput(0.75,0.25){+}\rput(1.25,0.25){1}\psline{->}(0.25,-0.5)(0.25,0)
         \begin{align*}
              A &\to BCDEF \\
              B &\to \varepsilon \\
              C &\to GH\\
              D &\to \varepsilon
         \end{align*}
     \item[Einsicht] Sei $A \to A \alpha\ |\ \beta$, wobei $\beta$ nicht mit $A$ beginnt.\\
         Es gilt $\texttt{FIRST}(A \alpha) \supseteq \texttt{FIRST}(\beta)$
             \begin{center}
             \begin{pspicture}(0,0)(2,2)
                 \rput[l](1,2){$\rnode{A}{A}a \overset{*}{\mapsto} \alpha \beta' \alpha$}
                 \psline[nodesep=3pt](A)(0.5,1.2)(1.5,1.2)(A)
                 \rput[l](1,1){$\rnode{B}{\beta}$}
                 \psline[nodesep=3pt](B)(0.3,0.2)(1.7,0.2)(B)
             \end{pspicture}
             \end{center}
     \item[Idee] Transformiere linksrekursive Grammatik $G$ in äquivalente Grammatik $G'$, so dass $\mathcal{L}(G) = \mathcal L(G')$ und 
                 $G'$ ist \emph{nicht} linksekursiv.
     \item[Prinzip] Eliminierung der Linksrekursion:
         \begin{description}
          \item Betrachte linksrekursiven Ausschnitt einer $G$. $A \to A\alpha\ |\ \beta$
          \item Welche Struktur haben Satzformen, die aus $A$ herleitbar sind?
          \begin{center}
              \pstree{\Tr*[name=root]{$A$}}{
                  \pstree{\Tr*{$A$}}{
                      \pstree{\Tr*{$A$}}{
                          \pstree{\Tr*[name=bla]{$A$}}{
                              \pstree{\Tr*{$A$}}{
                                   \Tfan{$\beta$}
                               }
                          }
                          \Tfan[nodesepB=3pt]{$\alpha$}
                      }
                  }
                  \Tfan[nodesepB=3pt]{$\alpha$}
              }
              \psbrace[rot=225,nodesepA=-2cm](root)(bla){$n$-mal $A \to A\alpha$}
              $\beta \alpha^n$
          \end{center}
          \item Ersetze diese beiden Regeln
              \begin{align*}
               A &\to \beta R \\
               R &\to \alpha R\ |\ \varepsilon
              \end{align*}
         \end{description}
     \item[Beispiel] Anwendung arithmetischer Ausdrücke (s. o.)
         \[ E \to E \alpha_1\ |\ E\alpha_2\ |\ \beta_1\ |\ \beta_2\]
         \begin{align*}
             E &\overset{*}{\mapsto} \beta_1 (\alpha_1\ |\ \alpha_2)^n \\
             E &\overset{*}{\mapsto} \beta_2 (\alpha_1\ |\ \alpha_2)^n
         \end{align*}
         \begin{align*}
             E &\to \beta_1 R\ |\ \beta_2 R \\
             R &\to \alpha_1 R\ |\ \alpha_2 R\ |\ \varepsilon
         \end{align*}
         konkret ($G'$):
         \framebox{
             \begin{minipage}{4cm}
                \begin{align*}
                    E &\to 0 R\ |\ 1 R \\
                    R &\to +0 R\ |\ +1R\ |\ \varepsilon
                \end{align*}
             \end{minipage}
         }
         \begin{center}
        \begin{minipage}{0.4\textwidth}
        \pstree{\Tr*{$E$}}{
            \pstree{\Tr*{$E$}}{
                \pstree{\Tr*{$E$}}{
                    \Tr*{1}
                }
                \Tr*{+}
                \Tr*{0}
            }
            \Tr*{+}
            \Tr*{0}
        }
        \end{minipage}
\begin{minipage}{0.4\textwidth}
\pstree{\Tr*{$E$}}{
    \Tr*{1}
    \pstree{\Tr*{$R$}}{
        \Tr*{+}
        \Tr*{0}
        \pstree{\Tr*{$R$}}{            
            \Tr*{+}
            \Tr*{1}
            \pstree{\Tr*{$R$}}{
                \Tr*{$\varepsilon$}
            }
        }
    }
}
\end{minipage}\end{center}
Übersetzungsschema schwierig bis nicht machbar
    \end{description}
    \begin{align*}
     G{:}\ E &\to E + T\ |\ E - T\ |\ T & G'{:}\ E &\to T R \\
     T &\to 0\ |\ 1 & R &\to + T R\ |\ - T R\ |\ \varepsilon \\
     && T &\to 0\ |\ 1
    \end{align*}
    Übersetzungsschema zu $G$:
    \begin{align*}
     E &\to E \pm T \{\texttt{print('+')}\}\ |\ T \\
     T &\to 0\{\texttt{print('0')}\ |\ 1\{\texttt{print('1')}\}\}
    \end{align*}
\item[Idee:] Eliminiere Linksrekursion aus dem Übersetzungsschema unter Einbeziehung der semantischen Aktionen.
\begin{align*}
 E &\to TR \\
  R &\to +T \{\texttt{print('+')}\} R\ |\ -T \{\texttt{print('-')}\} R\ |\ \varepsilon \\
  T &\to ...
\end{align*}
\end{description}
\section{Einfacher Übersetzer}
Prädiktive Syntaxanalyse angereichert um übernommenes Übersetzungsschema.
\begin{lstlisting}[language=Java]
// match-Prozedur, wie gesehen
void E() { T(); R(); }
void R() { 
    if (lookahead == '+') { match('+'); T(); print('+'); R; } 
    else if (lookahead == '-') { match('-'); T(); print('-'); R; } 
}
void T() {
    if (lookahead == '0') { match('0'); print('0'); }
    else if (lookahead == '1') { match('1'); print('1'); }
}
\end{lstlisting}

