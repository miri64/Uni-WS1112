\section{Übersetzungsschemata}
Syntax Directed Translation Schemes (SDT)
\Defi Eine kontextfreie Grammatik, in deren Produktionen auf der rechten Regelseite semantische Aktionen eingestreut sind, heißt \emph{Übersetzungsschema}, wobei semantische Aktionen, in geschweifte Klammern eingeschlossene Code-Fragmente der Implementierungssprache sind.
\begin{itemize}
 \item Semantische Aktionen tragen nicht zu der erzeugten Sprache bei. Beim Aufbau des Parsebaumes werden sie wie Terminale behandelt, jedoch mit gestrichelten Kanten zu deren Blättern (\emph{Pseudoterminale})
 \item Jedes Übersetzungsschema definiert eine Ausführung der Code-Fragmente, wie sie durch einen depth-first-left-to-right Durchlauf des Parsebaumes gegeben ist.
\end{itemize}
\paragraph*{Ziel:} nach Möglichkeit SDD $\rightarrow$ SDT. \\
2 Spezialfälle:
\begin{enumerate}
 \item LR-Grammatik uns $S$-Attributierung
 \item LL-Grammatik und $L$-Attributierung
\end{enumerate}
(Attributierung sind auf dem Parser-Stack möglich!)
\Bsp SDD:
\begin{center}
    \begin{tabular}{l|l}
        $L \to E$\verb!\n! & \verb!print(!$E.v$\verb!)! \\
        $E \to E_1 + T$ & $E.v = E_1.v + T.v$ \\
        $E \to T$ & $E.v = T.v$ \\
        $T \to T_1 * F$ & $T.v = T_1.v * F.v$ \\
        $T \to F$ & $T.v = F.v$\\
        $F \to d$ & $F.v = d.l$\\
    \end{tabular}
\end{center}
$S$-Attributierung.\\
Generiere äquivalentes Übersetzungsschema im Fall (1. s. o.) durch Anfügen von semantischen Aktionen nach jeder Produktion, die gene den semantischen Regeln entsprechen.
\paragraph*{} SDT zum gegebenen Beispiel:
\begin{align*}
    L &\to E\texttt{\textbackslash n} \{\texttt{print(}E.v\texttt{);}\} \\
    E &\to E_1 + T \{E.v = E_1.v + T.v; \}
\end{align*}
\begin{center}
    \psset{treesep=2.3cm}
    \def\dedge{\ncline[linestyle=dashed,linecolor=red]}
    \pstree{\Tr{$L$}}{
        \pstree{\Tr{$E$}\nput{0}{\pssucc}{\color{red}v = 7}}{
            \pstree{\Tr{$E$}\nput{0}{\pssucc}{\color{red}v = 3}}{
                \pstree{\Tr{$T$}\nput{0}{\pssucc}{\color{red}v = 3}}{
                    \pstree{\Tr{$F$}\nput{0}{\pssucc}{\color{red}v = 3}}{
                        \Tr{d}\nput{0}{\pssucc}{\color{red}l = 3}
                        \Tr[edge=\dedge]{\color{red}\{$F.v = d.l$\}}
                    }
                    \Tr[edge=\dedge]{...}
                    
                }
                \Tr[edge=\dedge]{\color{red}\{$E.v = T.v$\}}
            }
            \Tr{+}
            \pstree{\Tr{$E$}\nput{0}{\pssucc}{\color{red}v = 4}}{
                \pstree{\Tr{$T$}\nput{0}{\pssucc}{\color{red}v = 4}}{
                    \pstree{\Tr{$F$}\nput{0}{\pssucc}{\color{red}v = 4}}{
                        \Tr{d}\nput{0}{\pssucc}{\color{red}l = 4}
                        \Tr[edge=\dedge]{...}
                    }
                    \Tr[edge=\dedge]{...}
                }
                \Tr[edge=\dedge]{...}
            }
            \Tr[edge=\dedge]{\color{red}\{$E.v = E_1.v + T.v$\}}
        }
        \Tr{\texttt{\textbackslash n}}
        \Tr[edge=\dedge]{\color{red}\{\texttt{print ...}\}}
    }
\end{center}

\paragraph*{Idee zur Vermeidung des Aufbaus des Parsebaumes:}
Parse Stack:
\begin{center}
    \begin{tabular}{cc|c|c|l}\cline{1-4}
        ... & $x$   & $y$   & $z$   & top\\\cline{1-4}
            & $X.x$ & $Y.y$ & $Z.z$ & Attributwerte\\\cline{1-4}
    \end{tabular}
\end{center}
Aus den SDT, welche durch Anfügen der semantischen Aktionen entstanden ist, lässt sich ein äquivalentes SDT erzeugen, das auf dem Parser-Stack operiert!
\begin{align*}
    L &\to E\texttt{\textbackslash n} \{\texttt{print(stack[top-1]}.v\texttt{); top = top - 1;}\} \\
    E &\to E_1 + T \{\texttt{stack[top-2]}.v\texttt{ = stack[top-2]}.v\texttt{ + stack[top]}.v\texttt{; top = top - 2;}\} \\
    &\vdots
\end{align*}
\begin{center}
    \begin{tabular}{cc|c|c|}\hline
        ... & $E_1$   & $+$   & $T$\\\hline
            & 3 & $\cdot$ & 4 \\\hline
    \end{tabular}
\end{center}
Postfix-Übersetzungsschemata
\subsection{Übersetzungsschemata mit semantischen Aktionen innerhalb der Produktionen}
Sei $B \to X \{a\} Y$ eine Produktion des gegebenen Übersetzungsschemas.
\begin{itemize}
\item   Beim Bottom-Up-Parsen führe wir $\{a\}$ aus, sobald $X$ auf dem Stack erscheint.
\item   Beim Top-Down-Parsen führen wir $\{a\}$ aus, bevor $Y$ expandiert wird bzw. mit aktueller Eingabe gematcht wird
\end{itemize}
Nicht alle SDTs lassen sich beim Parsen implementieren
\Bsp Infix $\longleftarrow$ Präfix-Notation: $E \to \{\texttt{print($+$);}\} E_1 + T$
\paragraph*{Frage:} Lässt sich entscheiden, ob ein SDT auf dem Parser-Stack implementiert werden kann?
\paragraph*{Ja:}  Ersetze jede semantische Aktion durch ein neues Nichtterminal $M_1, ..., M_k$ und füge Produktionen \linebreak
$M_i \to \varepsilon, i \leq i \leq k$ hinzu. Erzeuge Parse-Tabelle.

\paragraph*{Implementierung von $L$-Attributierungen zu gegebenen $LL$-Grammatik}
\begin{enumerate}
 \item Schreibe semantische Aktionen zur Berechnung ererbter Attribute von $A$ unmittelbar vor das entsprechende Vorkommen von $A$
         \[B \to ... \{A.a = ...\} A ... \{B.b = ...\}\]
 \item Schreibe semantische Aktionen zur Berechnung von synthetisierten Attributen gant an das Ende der entsprechenden Produktionen
\end{enumerate}

\Bsp Ausschnitt aus Zwischencode-Erzeugung, Hier: Kontrollfluss
\begin{center}
    \begin{tabular}{l|p{10cm}}
        \textbf{Produktion} & \textbf{sem. Regeln} \\\hline
        $\vdots$ & \\
        $S \to \textbf{while}(C) S_1$ & $L1 = \text{new}(); L2 = \text{new}(); S_1.\text{next} = L1;$ \newline $ C.\text{false} = S.\text{next}; C.\text{true} = L2;$ \newline
        $S.\text{code} = \text{label} \| L1 \| C.\text{code} \| \text{label} \| L2 \| S_1.\text{code}$ \\ 
    \end{tabular}
\end{center}
$S \to \textbf{while}(\{L1 = \text{new}(); L2 = \text{new}(); C.\text{false} = S.\text{next}; C.\text{true} = L2;\}C)\{S_1.\text{next} = L1;\} S_1 \{S.\text{code} = ...\}$



