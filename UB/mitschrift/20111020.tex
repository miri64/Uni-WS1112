\chapter{Ein einfacher syntaxgerichteter Übersetzer (Frontend)}
\paragraph{Definition:} Ein Viertupel $G = (\mathcal{N}, \mathcal{T}, S, \mathcal{P})$ heißt \textbf{kontextfreie Grammatik}, wenn
\begin{enumerate}
 \renewcommand{\theenumi}{(\roman{enumi})}
 \item $\mathcal{N}$ ist ein Alphabet (endl. Menge) von \emph{Nichtterminalen}, 
 \item $\mathcal{T}$ ist ein Alphabet (endl. Menge) von \emph{Terminalen}, 
 \item $S \in \mathcal{N}$ ist ein \emph{Startsymbol},
 \item $\mathcal{P}$ ist eine endliche Menge von \emph{Produktionen} der Form $A \to \alpha$, wobei $A \in \mathcal{N}$ und $\alpha \in (\mathcal{N} \cup \mathcal{T})^*$
\end{enumerate}

\paragraph{Konventionen:} 
\begin{itemize}
 \item $\varepsilon$ bezeichnet das leere Wort.
 \item Der Kopf der ersten Produktion ist das Startsymbol
 \item Wenn nur $\mathcal P$ angegeben wird, lassen sich $\mathcal T$, $\mathcal N$ und $S$ daraus generieren.
\end{itemize}

\paragraph{Beispiel:}
\begin{minipage}[t]{0.2\linewidth}
\begin{align*}
 S &\to aSb \\
 A &\to \varepsilon \\
 B &\to SA
\end{align*}
\end{minipage}

\paragraph{Definition:} Sei $G = (\mathcal{N}, \mathcal{T}, S, \mathcal{P})$ eine kontextfreie Grammatik. 
	Ein Wort $\gamma$ lässt sich in einem Schritt bezüglich $G$ aus $\beta$ \textbf{herleiten}, wenn es Produktionen $A \to \alpha$ gibt mit $\beta = \alpha_1A\alpha_2$ und $\gamma = \alpha_1\alpha\alpha_2$.

Geschrieben:
\begin{minipage}[t]{0.5\linewidth}
\begin{align*}
 \beta &\xrightarrow[G]{} \gamma \\
 \alpha_1A\alpha_2 &\xrightarrow[G]{} \alpha_1\alpha\alpha_2 & \text{$\alpha_1$, $\alpha_2$ heißt Kontext}
\end{align*}
\end{minipage}
\begin{itemize}
 \item $\gamma$ lässt sich bzgl $G$ aus $\beta$ herleiten $\gamma \xrightarrow[G]{*} \beta$
\end{itemize}

\paragraph{Definition:}
$\mathcal{L}(G) := \{w | w \in \mathcal{T}^*\text{ mit } S  \xrightarrow[G]{*} w \}$
ist die \textbf{von $\boldsymbol{G}$ erzeugte Sprache}.

\paragraph{Definition:} Jeder Produktion wird ein elementarer \textbf{Parsebaum zugeordnet}.\\
$A \to X_1X_2\dots X_n$ wird zugeordnet:
\begin{minipage}[t]{0.5\linewidth}
 \pstree{\Tr*{$A$}}{
	\Tr*{$X_1$} \Tr*{$X_2$} \Tr*{$\dots$} \Tr*{$X_n$}
 }
\end{minipage}
Ein Baum mit inneren Knoten aus $\mathcal{N}$ heißt Parsebaum bezüglich $G$, genau dann wenn die Wurzel mit $S$ markiert ist und alle inneren Knoten zusammen mit ihren Kindern elementare Parsebäume bezüglich $G$ sind.

\paragraph{Beispiel:}
\begin{minipage}[t]{0.2\linewidth}
\begin{align*}
 \mathtt{expr} &\to \mathtt{expr} + \mathtt{digit}\ |\ \mathtt{expr} - \mathtt{digit}\ |\ \mathtt{digit} \\
 \mathtt{digit} &\to 0\ |\ 1\ |\ \dots\ |\ 9 \\
\end{align*}
\end{minipage}
\[9 - 5 + 2 \in \mathcal{L}(G)\]
$\mathtt{expr} \xrightarrow[G]{} \mathtt{expr} + \mathtt{digit} \xrightarrow[G]{} \mathtt{expr} + 2 \xrightarrow[G]{} \mathtt{expr} - \mathtt{digit} + 2 \dots 9-5+2$\\
oder $\mathtt{expr} \xrightarrow[G]{} \mathtt{expr} + \mathtt{digit} \xrightarrow[G]{} \mathtt{expr} - \mathtt{digit} + \mathtt{digit} \dots 9-5+2$
\begin{minipage}[t]{0.5\linewidth}
 \psset{levelsep=0.5cm}
 \pstree{\Tr*{\tt expr}}{
	\pstree{\Tr*{\tt expr}}{
		\pstree{\Tr*{\tt expr}}{
			\pstree{\Tr*{\tt digit}}{
				\Tr*{9}
			}
		}
		\Tr*{$-$}
		\pstree{\Tr*{\tt digit}}{
			\Tr*{5}
		}
	}
	\Tr*{$+$}
	\pstree{\Tr*{\tt digit}}{
		\Tr*{2}
	}
 }
\end{minipage}

\paragraph{Einsicht:} Jeder Parsebaum repräsentiert eine \textit{Menge} von Herleitungen.\\
Alternative Grammtik $G_m$:
\begin{align*}
 A &\to D\ |\ A+A\ |\ A-A \\
 D &\to 0\ |\ 1\ |\ \dots\ |\ 9 \\
\end{align*}
\paragraph{Lemma:} $\mathcal{L}(G) = \mathcal{L}(G_m)$\\
Es gibt $w \in \mathcal{L}(G_m)$, so dass es zwei verschiedene Parsebäume zu $w$ gibt.

\paragraph{Definition:} Eine Grammatik $G$ heißt \textbf{mehrdeutig}, wenn ein Wort $\alpha$ aus $S$ hergeleitet werden kann, so dass es zu $\alpha$ mehr als einen Parsebaum gibt.

Zusammen mit einer Tabelle aller Operatoren, geordnet nach Priorität in aufsteigender Reihenfolge und ihren Assoziativitäten, lässt sich eine \emph{eindeutige} Grammatik erzeugen, die äquivalent ist.

\paragraph{Verfahren:}
\begin{itemize}
 \item Ordne jedem Niveau ein eigenes Nichtterminal, sowie ein neues für die Grundbausteine der Grammatik (im Beispiel $\mathtt{factor}$)
	\begin{center}
	\begin{tabular}{c|l|l}
	\textbf{Ops} & \textbf{Assoziativität} & Nichtterminal \\
	\hline\hline
	$+-$ & links & \texttt{expr} \\
	$*/$ & links & \texttt{term} \\
	\hline
	&&\texttt{factor}
	\end{tabular}
	\end{center}
 \item Beginne von unten nach oben und ordne jeweils angemessene Produktionen hinzu
	\begin{align*}
		\mathtt{factor} &\to \mathtt{digit}\ |\ (\mathtt{expr}) \\
		\mathtt{term} &\to \mathtt{term} * \mathtt{factor}\ |\ \mathtt{term} / \mathtt{factor}\ |\ \mathtt{factor} \\
		\mathtt{expr} &\to \mathtt{expr} + \mathtt{term}\  |\  \mathtt{expr} - \mathtt{term}\ |\ \mathtt{term}
	\end{align*}
 \item Drehe das Regelwerk um:
	\begin{align*}
		\mathtt{expr} &\to \mathtt{expr} + \mathtt{term}\  |\  \mathtt{expr} - \mathtt{term}\ |\ \mathtt{term} \\
		\mathtt{term} &\to \mathtt{term} * \mathtt{factor}\ |\ \mathtt{term} / \mathtt{factor}\ |\ \mathtt{factor} \\
		\mathtt{factor} &\to \mathtt{digit}\ |\ (\mathtt{expr}) \\
	\end{align*}
\end{itemize}






