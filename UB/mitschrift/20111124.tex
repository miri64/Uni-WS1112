\chapter{Syntaxanalyse (Parser)}
\begin{itemize}
 \item kontextfreie Grammatiken bieten eine präzise, formale Spezifikation der grammatikalischen Struktur typischer Programmiersprachen; meist gegeben in Form von BNF und EBNF
 \item Gewisse \emph{eingeschränkte} Klassen kontextfreie Sprachen lassen sich effizient parsen und sogar Parsergeneratoreb schreiben.
 \item Grammatikalische Struktur der Programme bieten das \textit{Gerüst}, an dem der Übersetzungsprozess induktiv organisiert werden kann als auch systematische Fehlererkennung erlaubt.
 \item kontextfreie Struktur erlaubt \textit{inkrementelle Entwicklung}.
\end{itemize}
Nicht-kontextfreie Features werden abgesondert und einzeln regelhaft behandelt!
\begin{center}
\begin{psmatrix}[colsep=2cm,rowsep=0.5cm]
& & \framebox{Fehlerbehandlung}\\
Quellsprache & \framebox{Lexer} & \framebox{Parser} &
 \framebox{\begin{minipage}{1.5cm}
            weitere Phasen des Frontends
           \end{minipage}} & ...\\
& & \framebox{Symboltabelle}
\end{psmatrix}\ncline{<->}{1,3}{2,2}\ncline{<->}{1,3}{2,3}\ncline{<->}{1,3}{2,4}\ncline{2,1}{2,2}\ncline[offset=0.5em]{2,2}{2,3}\naput{Token}\ncline[offset=0.5em]{2,3}{2,2}\naput{getNextToken}
\ncline[linestyle=dashed]{2,3}{2,4}\naput{Parsebaum} \ncline{2,4}{2,5}\naput{Zwischen-}\nbput{code}
\ncline{<->}{3,3}{2,2}\ncline{<->}{3,3}{2,3}\ncline{<->}{3,3}{2,4}
\end{center}

\section{Drei Klassen von \emph{Syntaxanalysemethoden}}
\begin{itemize}
 \item universelle: jede kontextfreie Grammatik ist analysierbar ($O(n^3)$, \emph{nicht} akzeptabel)
 \item top-down: effiziente Teilklassen LL-Grammatiken
 \item bottom-up: effiziente Teilklassen LR-Grammatiken
\end{itemize}
L (Eingabe wird von \textbf{l}inks nach rechts gelesen) L (Linksherleitung)\\
L (Eingabe wird von \textbf{l}inks nach rechts gelesen) R (Rechtsherleitung, in umgekehrter Reihenfolge)
\[\text{LL} \subseteq \text{LR}\]
\begin{itemize}
 \item Parser für LL-Grammatiken sind leicht von Hand geschrieben
 \item Parser für LR-Grammatiken werden mit Parsergeneratoren generiert
\end{itemize}

\paragraph*{Beispielgrammatiken} (hier Ausdrücke):
\begin{align*}
 G_1{:} \quad E &\to E + E\ |\ E * E\ |\ \operatorname{id}\ |\ (E)
\end{align*}
Ist mehrdeutig, lässt sich \emph{nicht effektiv parsen}
\begin{align*}
 G_2{:} \quad E &\to E + T\ |\ T\\
 T &\to T * F\ |\ F\\
 F &\to \operatorname{id}\ |\ (E)
\end{align*}
\begin{itemize}
 \item eindeutig
 \item ist LR-Grammatik, d. h. effizient bottom-up zu parsen
\end{itemize}
\begin{align*}
 G_3{:} \quad E &\to TE'\\
 E' &\to + TE'\ |\ \varepsilon\\
 T &\to FT' \\
 T' &\to *FT'\ |\ \varepsilon\\
 F &\to \operatorname{id}\ |\ (E)
\end{align*}
\begin{itemize}
 \item eindeutig
 \item effizient zu parsen
 \item LL-Grammatik
\end{itemize}

\section{Fehlerbehandlung}
\paragraph*{Kriterien}
\begin{itemize}
 \item Präzision (Lokalisierung, Beschreibung)
 \item Qualität (Erkenngun möglichst aller Fehler, schnelle Erholung)
 \item Effizienz
\end{itemize}

\paragraph*{Typische Methoden}
\begin{itemize}
 \item \textbf{Konstruktorientierte (lokale) Methoden:}
  \begin{itemize}
  \item lokale Manipulation beim Auftreten eines Fehlers, z. B. Vertauschen zweier Symbole, Ersetzen eines ein anderes, Weglassen... \item effizient und in typischen Fällen zielführend
  \end{itemize}
 \item \textbf{Panic Mode (Panische Fehlererholung):}
  \begin{itemize}
  \item Streiche alle Symole bis man auf ein Synchronisationssymbol stößt, typische Trenn- und Abschlusssymbol, unter anderen z. B. "`;"'
  \item besonders effizient (überspringt weitere Fehler in fehlerhafter Teilstruktur)
  \end{itemize}
 \item \textbf{Globale Korrektur}
  \begin{itemize}
  \item Manipulation der Tokenfolge durch minimale Anzahl elementarer Operationen; wie
   \begin{itemize}
   \item weglassen
   \item vertauschen
   \item einfügen
   \item ersetzen
   \end{itemize}
  \item Optimal hinsichtlich Qualität; Zyklen möglich
  \end{itemize}
 \item \textbf{Fehlerproduktionen}
  \Bsp $\texttt{stmt} \to \textbf{if}\ \texttt{expr}\ \textbf{then}\ \texttt{stmt}$ (Pascal) \\
    Fehlerproduktion: $\texttt{stmt} \to \textbf{if}\ \texttt{expr}\ \texttt{stmt}$ (Java)
    \begin{itemize}
     \item Reperatur einfach möglich durch Einfügen von \textbf{then}
    \end{itemize}
    Sehr präzise, effizient, nicht allgemein verwendbar
 \item \textbf{Mehrdeutigkeit}
  \begin{itemize}
  \item gewisse Mehrdeutigkeit in der Grammatik enthalten, die durch zusätzliche Regeln aufgelöst wird\\
  \paragraph*{Typisches Beispiel:} \emph{Dangling-Else}
    \begin{verbatim}
S -> if E then S          // bedingte Anweisung
  |  if E then S else S   // Verzweigung
  |  other
    \end{verbatim}
    Methode (fast immer möglich) besteht aus Transformation der Grammatik
    \begin{verbatim}
S -> M                    // (matched-Anwesiung, geschlossen)
  | U                     // (unmatched Anweisung, offen)
M -> if E then M else M
  |  other
U -> if E then S
  |  if E then M else U
    \end{verbatim}
    Eindeutigkeit
  \end{itemize}
\end{itemize}


