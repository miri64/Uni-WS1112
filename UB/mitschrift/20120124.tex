\section{Implementierung von Drei-Adress-Code}
\Bsp $x = y \operatorname{op} z \Rightarrow$ \emph{Quadrupeldarstellung}
\begin{itemize}
 \item Jeder Befehl hat die Struktur $\operatorname{op} \operatorname{arg}_1 \operatorname{arg}_2 \operatorname{result}$
 \item Unäre Operatoren machen keinen Gebrauch von $\operatorname{arg}_2$
 \item Kopierbefehle machen keinen Gebrauch von $\operatorname{arg}_2$
 \item Sprungbefehle schreiben Ziel in result
\end{itemize}
\Bsp 
\begin{align*}
 a &= b * -c + b * -c \\
 t_1 &= \operatorname{minus} c \\
 t_2 &= b * t_1 \\
 t_3 &= \operatorname{minus} c \\
 t_4 &= b * t_3 \\
 t_5 &= t_2 + t_4 \\
 a &= t_5
\end{align*}
Quadrupeldarstellung:
\begin{center}
    \begin{tabular}{r|c|c|c|c}
          & op    & $\operatorname{arg}_1$ & $\operatorname{arg}_2$ & result \\\hline
        0 & minus & $c$                    &                        & $t_1$  \\\hline
        1 & $*$   & $b$                    & $t_1$                  & $t_2$  \\\hline
        2 & minus & $c$                    &                        & $t_3$  \\\hline
        3 & $*$   & $b$                    & $t_3$                  & $t_4$  \\\hline
        4 & $+$   & $t_2$                  & $t_4$                  & $t_5$  \\\hline
        1 & $=$   & $t_5$                  &                        & $a$    \\\hline
    \end{tabular}
\end{center}

Platzminimierung $\rightarrow$ Tripeldarstellung \hspace{0.4\linewidth}\rnode{ref0_label}{Veweis auf 0-ten Befehl}
\begin{center}
    \begin{tabular}{r|c|c|c}
          & op    & $\operatorname{arg}_1$ & $\operatorname{arg}_2$ \\\hline
        0 & minus & $c$                    &                        \\\hline
        1 & $*$   & $b$                    & \rnode{ref0}{$(0)$}    \\\hline
        2 & minus & $c$                    &                        \\\hline
        3 & $*$   & $b$                    & $(2)$                  \\\hline
        4 & $+$   & $(1)$                  & $(3)$                  \\\hline
        1 & $=$   & $a$                    & $(4)$                  \\\hline
    \end{tabular}\ncline{->}{ref0_label}{ref0}
\end{center}
\textbf{Achtung:} Einige Befehle verwenden drei Adressen, sie werden durch je zwei Befehle dargestellt.
\[x[i] = y\]
Nachteil: \emph{Verschiebbarkeit}.
\paragraph*{Indirekte Tripel}
\begin{center}
    \begin{tabular}{r|c|c|c}
        33 & (0) \\\hline
        34 & (1) \\\hline
        35 & (2) \\\hline
        36 & (3) \\\hline
        37 & (4) \\\hline
        38 & (5) \\\hline
    \end{tabular}\ncline{->}{ref0_label}{ref0}
\end{center}

\section{Static-Single-Assignment (SSA)}
\Bsp
\begin{align*}
 p &= a+b & p_1 &= a+b \\
 q &= p-c & q_1 &= p_1-c\\
 p &= q*d & p_2 &= q_1*d\\
 p &= e-p & p_3 &= r-p_3\\
 q &= p+q & q_2 &= p_3+q_1
\end{align*}
Problem: \verb!if (flag) x = 0; else x = 1;! $\Rightarrow$ \verb!y = 3 * !$\underset{?}{\mathtt{x}}$\verb!;!\\
Neuer Operator $\Phi$ $\Rightarrow x_3 = \Phi(x_1,x_2)$

\section{Typen und Deklarationen}
\begin{itemize}
 \item Typüberprüfungen wichtig bei Fehlererkennung
 \item Typisierung dient effizienter Speicherung
\end{itemize}
Modernes Typkonzept enthält induktive Strukturen, d. h. Typausdrücke zur Definition von Typen.
\Defi (Typausdrücke, $\mathcal{TA}$)
\begin{enumerate}
 \item Elementare Typen
 \begin{itemize}
  \item Basistypen, z. B.: \textbf{float}, \textbf{int}, ...
  \item Typvariablen - Bezeichner
 \end{itemize}
 \item Mit $t \in \mathcal{TA}$ und $n \in \mathbb{N}$ ist auch \texttt{array($n$,$t$)} ein Typ.
 \item Mit Bezeichnern $f_1, ..., f_n$ und Typen $t_1, ..., t_n$ ist auch \texttt{record($f_1 t_1$, $...$, $f_n t_n$)} $\in \mathcal{TA}$
 \item Mit Typausdrücken $s$ und $t$ ist auch $s \to t \in \mathcal{TA}$
 \item Mit $s, t \in \mathcal{TA}$ ist auch $s \times t \in \mathcal{TA}$
\end{enumerate}

\Bsp zur Berechnung der Breite von Werten des Typs \texttt{array ....} \\
Auszug aus Syntax der Quellsprache zur Deklaration von Arrays
\begin{align*}
    T &\to BC \\
    B &\to \texttt{int}\ |\ \texttt{float} \\
    C &\to \texttt{[}\textit{num}\texttt{]}C\ |\ \varepsilon
\end{align*}
z. B. \texttt{int[2][3]}
\begin{center}
    \pstree{\Tr{\texttt{array}}}{
        \Tr{2}
        \pstree{\Tr{\texttt{array}}}{
            \Tr{3}
            \Tr{\texttt{int}}
        }
    }
\end{center}

SDD:
\begin{center}
    \psset{treesep=2cm}
    \pstree{\Tr{$T$}}{
        \pstree{\Tr{$B$}}{
            \Tr{\texttt{int}}
        }
        \pstree{\Tr{$C$}}{
            \Tr{\texttt{[}}
            \Tr{\textit{num}}\nput{0}{\pssucc}{\color{red} v = 2}
            \Tr{\texttt{]}}
            \pstree{\Tr{$C$}}{
                \Tr{\texttt{[}}
                \Tr{\textit{num}}\nput{0}{\pssucc}{\color{red} v = 2}
                \Tr{\texttt{]}}
                \pstree{\Tr{$C$}}{\Tr{$\varepsilon$}}
            }
        }
    }
\end{center}


