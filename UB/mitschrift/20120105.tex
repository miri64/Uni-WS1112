\section{Zusammenfassung}
Aufbau eines SLR-Parsers (simple LR(1)-Parser)
\paragraph*{Struktur}
\begin{center}
    \begin{psmatrix}
        & Eingabe: \begin{tabular}{|c|c|c|c|c|c|}\hline $a_1$ & ... & \rnode{a_i}{$a_i$} & ... & $a_n$ & \$ \\\hline\end{tabular} $w \in G$ & \\
        Stack \begin{tabular}{|c|}$S_n$ \\\hline $\vdots$ \\\hline $S_1$ \\\hline 0 \\\hline\end{tabular} & \framebox{Parse-Programm}
        & \begin{minipage}{5cm}Folge von Produktionen, umgekehrte Rechtsherleitung\end{minipage} \\
        & \begin{tabular}{c|c|c|}\cline{2-3} 0 & \multirow{3}{*}{\rnode{ACTION}{\Large ACTION}} & \multirow{3}{*}{\rnode{GOTO}{\Large GOTO}} \\
                                           $\vdots$ & & \\
                                           $k$ & & \\\cline{2-3}
       \end{tabular}

    \end{psmatrix}
    \psset{arrows=->}
    \ncline{2,2}{a_i}\ncline{2,2}{2,1}\ncline{2,2}{2,3}\naput{Ausgabe}
    \ncline{2,2}{ACTION}\ncline{2,2}{GOTO}
\end{center}
Elemente auf dem Stack sind Nummern der Itemmenge aus der kanonischen LR(0)-Item-Sammlung. Zu jeder Item-Menge gehört genau ein Grammatiksymbol.\\
Die Steuerung erfolgt über eine Parse-Tabelle, deren Zeilen mit Zustandsnummern eines LR(0)-Automaten markiert sind. Der ACTION-Teil hat Spalten mit Terminalsymbolen und $\$$ und der GOTO-Teil hat Spalten mit Nichtterminalsymbolen. Die Einträge der Parse-Tabelle werden wie folgt bestimmt
\begin{enumerate}
 \item ACTION$[i,a]$ erhält "`shift $I_j$"' (kurz s$j$), wenn $[A \to \alpha.a\beta] \in I_i$ und GOTO$(I_i,a) = I_j$.
 \item ACTION$[i,a]$ erhält "`reduce $A \to \alpha$"' (kurz r$j$), wenn $[A \to \alpha.] \in I_i, A \neq S'$ und $a \in \text{FOLLOW}(A)$. Beachte: $j$ ist die Nummer der Produktion $A \to \alpha.$
 \item ACTION$[i,\$]$ erhält "`accept"' (kurz acc), wenn $[S' \to S.] \in I_i$.
 \item GOTO$[i,A]$ erhält den Eintrag $j$, wenn GOTO$(I_i, A) = I_j$
 \item Leere Felder werden als Fehler interpretiert.
\end{enumerate}
\begin{itemize}
 \item Mehrfacheinträge im ACTION-Teil, die durch Regeln 1. und 2. entstehen können, signalisieren, dass die Grammatik \emph{nicht} SLR(1) ist
\end{itemize}
\paragraph*{Parse-Tabelle für einfache Ausdrücke}
\begin{align*}
 1. E &\to E + T & \text{FOLLOW}(E) &= \{+,),\$\} \\
 2. E &\to T     & \text{FOLLOW}(T) &= \{+,*,),\$\} \\
 3. T &\to T * F & \text{FOLLOW}(F) &= \{+,*,),\$\} \\
 4. T &\to F   \\
 5. F &\to (E) \\
 6. F &\to i
\end{align*}
\begin{center}
    \begin{tabular}{c|c|c|c|c|c|c||c|c|c|}
            & $i$ & $+$ & $*$ & $($ & $)$  & $\$$ & $E$ & $T$ & $F$ \\\hline\hline
        0   & s5  &     &     & s4  &      &      & 1   & 2   & 3   \\\hline
        1   &     & s6  &     &     &      & acc  &     &     &     \\\hline
        2   &     & r2  & s7  &     & r2   & r2   &     &     &     \\\hline
        3   &     & r4  & r4  &     & r4   & r4   &     &     &     \\\hline
        4   & s5  &     & s4  &     &      &      & 8   & 2   & 3   \\\hline
        5   &     & r6  & r6  &     & r6   & r6   &     &     &     \\\hline
        6   & s5  &     & s4  &     &      &      &     & 9   & 3   \\\hline
        7   & s5  &     & s4  &     &      &      &     &     & 10  \\\hline
        8   &     & s6  &     &     & s11  &      &     &     &     \\\hline
        9   &     & r1  & s7  &     & r1   & r1   &     &     &     \\\hline
        10  &     & r3  & r3  &     & r3   & r3   &     &     &     \\\hline
        11  &     & r5  & r5  &     & r5   & r5   &     &     &     \\\hline
    \end{tabular}
\end{center}

\paragraph*{Konfiguration des LR-Parsers}
Der Anfangszustand ist $(0, a_1...a_n\$)$. Der jeweilige Folgezustand zu$(s_0...s_m, a_ia_{i+1}...a_n\$)$  berechnet sich wie folgt:
\begin{enumerate}
 \item Wenn ACTION$[s_m,a_i] = \text{s}j$, dann setze Folgezustand auf $(s_0...s_mj, a_{i+1}...a_n\$)$. Beachte: $a_i$ gehört zu $I_j$
 \item Denn ACTION$[s_m,a_i] = \text{r}j$ und die $j$-te Produktion die Form $A \to X_1 ... X_k$ und GOTO$[s_m, A] = t$ ist, dann setze Folgezustand auf $(s_0...s_{m-k}t, a_i...a_n\$)$.
 \item Wenn ACTION$[s_m,\$] = \text{acc}$, dann signalisiere erfolgreiches Parsen.
 \item Wenn ACTION$[s_m,a_i]$ leer ist, dann rufe Fehlerbehandlung auf.
\end{enumerate}

\paragraph*{Brauchbarer Präfix}
Im Keller steht immer eine Präfix einer Rechtssatzform. Die Konfiguration enthält immer eine Rechtssatzform (unter Konkatination).
\Defi (Brauchbarer Präfix) Ein Präfix einer Rechtssatzform erstreckt sich nicht über den Handle hinaus.
\Lemma SLR-Parser erkennen brauchbare Präfixe.