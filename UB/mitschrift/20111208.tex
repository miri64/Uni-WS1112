\section{Bottom-Up-Syntaxanalyse}
\subsection{Einführung}
Naives Voegehen am Beispiel einfacher artihmetischer Ausdrücke
\begin{align*}
 E &\to E+T\ |\ T \tag{1. / 2.} \\
 T &\to T*F\ |\ F \tag{3. / 4.}\\
 F &\to (E)\ |\ i \tag{5. / 6.}
\end{align*}
\[                        \underline{i} * i 
\underset{6}{\rightarrow} \underline{F} * i 
\underset{4}{\rightarrow} T * \underline{i}
\underset{6}{\rightarrow} \underline{T * F}
\underset{3}{\rightarrow} \underline{T}
\underset{2}{\rightarrow} E
\]
in umgekehrter Reihenfolge ergibt sich eine Rechtsherleitung des analysierten Wortes. \\
alternativ:
\[                        \underline{T} * i
\underset{2}{\rightarrow} \underline{E} * i
\underset{6}{\rightarrow} E * \underline{F}
\underset{4}{\rightarrow} E * \underline{T}
\underset{2}{\rightarrow} E * E \tag{Sackgasse!}
\]
\paragraph*{Informelle Definition:} Ein \emph{Handle} in einer umgekehrten Rechtsherleitung ist der Teilstring $\beta$ der aktuellen Rechtssatzform, der als nächstes mit der Regel $A \to \beta$ auf $A$ zu reduzieren ist.
\paragraph*{Formale Definition:} Sei $S \underset{rm}{\xrightarrow{*}} \alpha Aw \underset{rm}{\rightarrow} \alpha \beta w$ eine Rechtsherleitung. Die Produktion $A \to \beta$ zusammen mit ihrem Vorkommen nach $\alpha$ heißt \emph{Handle} von $\alpha \beta w$
\Lemma Wenn $G$ eindeutig ist, dann gibt es zu jeder Rechtssatzform genau einen Handle.
\paragraph*{Konvention:} Wenn es zu $\beta$ nur eine rechte Regelseite der Form $\beta$ gibt und $\beta$ nur einmal in $\alpha \beta w$ vorkommt, dann spricht man: "`$\beta$ ist Handle von $\alpha \beta w$."'

\subsection{Bottom-Up-Syntaxanalyse durch Handle-Produktionen}
Sei $G$ eine kontextfreie Grammatik und $w \in \mathcal{L}(G)$. Sei $S = \gamma_0 \to \gamma_1 \to ... \gamma_n = w$ eine Rechtsherleitung von $w$.
\begin{enumerate}
 \item Beginne mit $i = n$
 \item \emph{Lokalisiere den Handle} in $\gamma_i$ und ersetze ihn durch das entsprechende Nichtterminal, um $\gamma_{i-1}$ zu erhalten.
 \item Wiederhole 2. bis man auf $\gamma_0 = S$ stößt.
\end{enumerate}
\paragraph*{Problem:} Wie lokalisiert man den Handle? (später)\\

\subsection{Stack-Implementierung der Bottom-Up-Analyse (Shift-Reduce-Parser)}
\begin{itemize}
 \item Verwende einen Stack, der Symbole $\mathcal{N} \cup \mathcal{T}$ enthält sowie \$ am Grund
 \item Eingabe $w \in \mathcal{T}^*$, mit $w\$$ auf Input (hier von links nach rechts)
 \item Schreibe top von Stack jeweils nach rechts.
 \item Anfangszustand: Stack: $\$$; Input: $w\$$
\end{itemize}

\paragraph*{Verfahren:}
\begin{itemize}
 \item Schreibe (shift) so lange Symbole vom Input auf den Stack, bis der Handle auf der Kellerspitze erscheint.
 \item Reduziere (reduce) am Handle auf der Kellerspitze
 \item Wiederhole diesen Prozess, bis entweder $\$S$ auf dem Keller erscheint mit Input $= \$$m dann akzeptiere $w$; oder ein Fehler auftritt, der gemeldet wird.
\end{itemize}

\begin{center}
 \begin{tabular}{|c|c|c|c|}
  Stack   & Input   & Aktion & Ausgabe     \\\hline
  $\$$    & $i*i\$$ & shift  &             \\
  $\$i$   & $*i\$$  & reduce & $F \to i$   \\
  $\$F$   & $*i\$$  & reduce & $T \to F$   \\
  $\$T$   & $*i\$$  & shift  &             \\
  $\$T*$  & $i\$$   & shift  &             \\
  $\$T*i$ & $\$$    & reduce & $F \to i$   \\
  $\$T*F$ & $\$$    & reduce & $T \to T*F$ \\
  $\$T$   & $\$$    & reduce & $E \to T$   \\
  $\$E$   & $\$$    & accept &
 \end{tabular}
\end{center}
Parsebaum:
\begin{center}
\pstree{\Tr{$E$}}{
    \pstree{\Tr{$T$}}{
        \pstree{\Tr{$T$}}{
            \pstree{\Tr{$F$}}{
                \Tr{$i$}
            }
        }
        \Tr{*}
        \pstree{\Tr{$F$}}{
            \Tr{$i$}
        }
    }
}
\end{center}
Nichtdeterministisch
\paragraph*{Konflikte:}
\begin{itemize}
 \item \emph{shift-reduce}-Konflikte
 \item \emph{reduce-reduce}-Konflikte
\end{itemize}

\subsection{Hierarchie innerhalb der Klasse der kontextfreien Sprachen}
\begin{description}
 \item[DPDA] Deterministischer Kellerautomat
 \item[NPDA] Nicht-deterministischer Kellerautomat
\end{description}
\[\mathcal{L}(\text{LL(1)}) \subsetneq \mathcal{L}(\text{DPDA}) \subsetneq \mathcal{L}(\text{DPDA}) = \text{alle kontextfreien Sprachen}\]
\[\text{LL(1)} \subseteq \text{LL($k$)}, \mathcal{L}(\text{LL(1)}) \subsetneq \mathcal{L}(\text{LL(2)}) ... \subsetneq \mathcal{L}(\text{LL($k$)}) \subsetneq \mathcal{L}(\text{DPDA})\]
\Defi Grammatik heißt \emph{LR(1)-Grammatik}, genau dann wenn es einen deterministischen Kellerautomaten gibt, der mit Lookahead $k$ eine Rechtsherleitung der eingabe in umgekehrter Reihenfolge liefert.
\Satz LR(1) ist mächtiger als $\textrm{LL} = \bigcup\limits_{k \in \mathbb{N}} \textrm{LL}(k)$



