\section{Top-Down-Analyse}
\paragraph{allgemeines Verfahren} (für beliebige kontextfreie Grammatik $G = (\mathcal{N}, \mathcal{T}, S, \mathcal{P})$). Schreibe zu jedem $X$ aus $\mathcal{N}$ eine Methode wie folgt:
\begin{lstlisting}[language=Java,mathescape=true,morekeywords={to}]
void $X$() {
    waehle eine $X$-Produktion $X \to X_1...X_n$ aus // nichtdeterministisch 
    for ($i$ = 1 to $n$) {
        if ($X_i \in \mathcal{N}$) $X_i$();           // rofe $X_i$-Methode rekursiv auf
        else if ($X_i$ == input) input = nextToken();
        else Fehlermeldung/-behandlung
    }
}

// Top-Down-Parser
void parser() {
    Token input = nextToken();
    $S$();
}
\end{lstlisting}

\begin{itemize}
 \item \emph{nichtdeterministisch}: Inakzeptabel hinsichtlich Effizienz\\
         $\Rightarrow$ Löse Nichtdeterminismus auf!
 \item Auflösung von Nichtdeterminismen nicht immer möglich, aber in der Praxis in der Regel doch
 \item Rezept: Entscheide anhand des aktuellen Inputs (Lookahead)
\end{itemize}
\Bsp
\begin{align*}
 E &\to TE'\\
 E' &\to +TE'\ |\ \varepsilon \\
 T &\to FT' \\
 T' &\to *FT'\ |\ \varepsilon \\
 F &\to (E)\ |\ \operatorname{id}
\end{align*}
\psset{arrows=-}
\begin{description}
 \item[Eingabe] \texttt{id + id * id}
 \item[Syntaxanalyse]
 \begin{center}
    \pstree{\Tr*{$E$}}{
        \pstree{\Tr*{$T$}}{
            \pstree{\Tr*{$F$}}{
                \Tr*{id}
            }
            \pstree{\Tr*{$T'$}}{
                \Tr*{$\varepsilon$}
            }
        }
        \pstree{\Tr*{$E'$}}{
            \Tr*{+}
            \pstree{\Tr*{$T$}}{
                \pstree{\Tr*{$F$}}{
                    \Tr*{id}
                }
                \pstree{\Tr*{$T'$}}{
                    \Tr*{*}
                    \pstree{\Tr*{$F$}}{
                        \Tr*{id}
                    }
                    \pstree{\Tr*{$T'$}}{
                        \Tr*{$\varepsilon$}
                    }
                }
            }
            \Tr*{$E$}
        }
    }
 \end{center}
\end{description}
\subsection{Eliminiertung der Linksrekursion}
\begin{itemize}
 \item geht immer!
 \item Bisher: (direkte Linksrekursion) $A \to A \alpha\ |\ \beta \quad \rightsquigarrow \quad A \to \beta A'; A' \to \alpha A'\ |\ \varepsilon$
 \item mehrere liksrekursive Produktionen: $A \to A\alpha_1\ |\ ...\ |\ A\alpha_n\ |\ \beta_1\ |\ ...\ |\ \beta_m$
         \begin{align*}
          \rightsquigarrow \quad   A &\to \beta_1A'\ |\ ...\ |\ \beta_mA' \\
                              A' &\to \alpha_1A'\ |\ ...\ |\ \alpha_nA'\ |\ \varepsilon
         \end{align*}
\end{itemize}
\Bsp
\begin{align*}
 S &\to Aa\ |\ b \\
 A &\to Ac\ |\ Sd\ |\ \varepsilon
\end{align*}
\begin{description}
 \item[Eingabe:] $bda\$$
 \item[indirekte Linksrekursion:] $\underline{S} \to \underline{A}a \to Sda...$
     \begin{center}
         \pstree{\Tr{$S$}}{
             \pstree{\Tr{$A$}}{
                 \Tr{$S$}
                 \Tr{$d$}
             }
             \Tr{a}
         }
     \end{center}
\end{description}

\subsubsection{Allgemeines Verfahren zur Eliminierung der Linksrekursion}
\begin{lstlisting}[language=Java,mathescape=true,morekeywords={to}]
Sortiere alle Nichtdeterminimale $A_1...A_n$
for ($i$ = 1 to $k$) {
    for ($j$ = 1 to $i-1$) {
        Ersetze jede Produktion $A_i \to A_j\gamma$ durch $A \to \delta_1\gamma\ |\ ...\ |\ \delta_n\gamma$,
                wobei $A_j \to \delta_1\ |\ ...\ |\ \delta_n$ alle $A_j$-Produktionen sind
    }
    Eliminiere direkte Linksrekursion aus $A_i$-Prod.
}
\end{lstlisting}

\Bsp $S = A_1, \quad A = A_2$
\begin{description}
 \item $i = 1 \qquad \checkmark$
 \item $i = 2 \qquad$ Ergebnis:
     \begin{align*}
      S &\to Aa\ |\ b
      A &\to Ac\ |\ Aad\ |\ bd \ |\ \varepsilon
     \end{align*}
 \item Eliminiere direkte Linksrekursion aus $A_i$-Prod.
     \begin{align*}
      S &\to Aa\ |\ b \\
      A &\to bdA'\ |\ A' \\
      A &\to cA'\ |\ adA'\ |\ \varepsilon
     \end{align*}
\end{description}

\subsection{Linksfaktorisierung}
\begin{itemize}
 \item Vermeiden gemeinsamer, nichttrivialer Präfixe
 \Bsp $S \to \textbf{if}\ E\ \textbf{then}\ S\ |\ \textbf{if}\ E\ \textbf{then}\ S\ \textbf{else} S\ |\ ...$
 \paragraph*{Lösung:} 
     \begin{align*}
          S &\to \textbf{if}\ E\ \textbf{then} SS' \\
         S' &\to \textbf{else}\ S\ |\ \varepsilon
     \end{align*}
 \paragraph*{allg. Verfahren:} Bestimme zu jedem $A$ den längsten Präfix $\alpha$, der in zwei oder mehr Produktionen zu $A$ vorkommt.
    Falls $\alpha \neq \varepsilon$, dann ersetzen wir $A \to \alpha\beta_1\ |\ ...\ |\ \alpha\beta_n\ |\ \gamma_1\ |\ ...\ |\ \gamma_m$ durch
    \begin{align*}
     A &\to \alpha A'\ |\ \gamma_1\ |\ ...\ |\ \gamma_m \\
     A' &\to \beta_1\ |\ ...\ |\ \beta_n
    \end{align*}
    Erschöpfende Anwendungen
\end{itemize}

\subsection{Behandlung der $\varepsilon$-Produktionen}
\begin{itemize}
 \item $A \to a\ |\ \beta$
 \item $a \in \operatorname{FIRST}(\alpha)$? $A \to \alpha$ 
 \item $\varepsilon \in \operatorname{FIRST}(\beta)$ \\
       $\Rightarrow$ Lookahead steht hinter einem "`leeren Wort"'
 \item $\operatorname{FIRST}(\alpha) = \{a\ |\ \exists\beta \in \mathcal{T} \cup \mathcal{N} \cup \varepsilon{:}\ \alpha \xrightarrow{*} a\beta\} \cup \{\varepsilon\ |\ \alpha \xrightarrow{*} \varepsilon\}$\\
       $\operatorname{FOLLOW}(A) = \{a\ |\ \exists\alpha, \beta \in \mathcal{T} \cup \mathcal{N} \cup \varepsilon{:}\ S \to \alpha Aa\beta\} \cup \{\$\ |\ \exists\alpha{:}\ S \xrightarrow{*} \alpha A\}$\\
       \emph{formale Definition}
\end{itemize}

\subsubsection{Verfahren zur Berechnung der FIRST- und FOLLOW-Mengen}
\begin{itemize}
 \item FIRST-Mengen zu allen rechten Regelseiten
 \item FOLLOW-Mengen zu allen Nichtterminalen
 \item Wende folgende Regel an, bis keine neuen  Elemente hinzutreten
     \begin{enumerate}
      \item $\forall a \in \mathcal{T}{:}\ FIRST(a) = \{a\}$
      \item Für jede Produktion $A \to X_1...X_n$ mit $n \geq 1$ und jedes $1 \leq i < n$ füge $\operatorname{FIRST}(X_i)$ zu $\operatorname{FIRST}(A)$ hinzu, falls alle $\operatorname{FIRST}(X_j)$ das Element $\varepsilon$ enthalten mit $1 \leq j < i$.
      Füge $\varepsilon$ zu $\operatorname{FIRST}(A)$ hinzu, falls alle $\operatorname{FIRST}(X_i)$ das Element $\varepsilon$ enthalten ($1 \leq i \leq n$)
      \item Füge $\varepsilon$ zu $\operatorname{FIRST}(A)$, falls $A \to \varepsilon$ in $\mathcal{P}$
     \end{enumerate}
     Zur Bestimmung von $\operatorname{FIRST}(\alpha)$ zu jeder rechten Regelseite, verfahre analog zu Regel 2 $\alpha = X_1...X_n$
  \Bsp  $\mathcal{T} = \{+,*,(,),\textbf{id}\}$,
        $\mathcal{N} = \{E, E', T, T', F\}$
        \begin{align*}
         \operatorname{FIRST}(E) &= \{{\color{blue}(, \textbf{id}}\} \\
         \operatorname{FIRST}(E') &= \{+, \varepsilon\} \\
         \operatorname{FIRST}(T) &= \{{\color{red}(, \textbf{id}}\} \\
         \operatorname{FIRST}(T') &= \{*, \varepsilon\} \\
         \operatorname{FIRST}(F) &= \{(, \textbf{id}\}
        \end{align*}
        (1. Durchlauf, {\color{red}2. Durchlauf}, {\color{blue}3. Durchlauf})
        \begin{align*}
         \operatorname{FIRST}(+TE') &= \{+\}  & \operatorname{FIRST}(\varepsilon) &= \{\varepsilon\} \\
         \operatorname{FIRST}(*FT') &= \{*\} \\
         \operatorname{FIRST}((E)) &= \{(\}   & \operatorname{FIRST}(\textbf{id}) &= \{\textbf{id}\}
        \end{align*}
 \item FOLLOW-Generieriung ("`Erschöpfende Anwendung folgender Regeln"'):
     \begin{enumerate}
      \item Füge $\$$ zu $\operatorname{FOLLOW}(S)$ hinzu.
      \item Für jede Produktion der Form $A \to \alpha B \beta$ füge $\operatorname{FIRST}(\beta) \setminus \{\varepsilon\}$ zu $\operatorname{FOLLOW}(B)$ hinzu.
      \item Für jede Produktion der Form $A \to \alpha B$ oder $A \to \alpha B \beta$ mit $\varepsilon \in \operatorname{FIRST}(\beta)$ füge $\operatorname{FOLLOW}(A)$ zu $\operatorname{FOLLOW}(B)$ hinzu
     \end{enumerate}
  \Bsp  \begin{align*}
         \operatorname{FOLLOW}(E) &= \{\$,)\} \\
         \operatorname{FOLLOW}(E') &= \{\$,)\} \\
         \operatorname{FOLLOW}(T) &= \{+,\$, )\} \\
         \operatorname{FOLLOW}(T') &= \{+, \$, )\} \\
         \operatorname{FOLLOW}(F) &= \{*, +, \$, )\}
        \end{align*}
\end{itemize}
\Defi Eine kontextfreie Grammatik $G = (\mathcal{N},\mathcal{T},S,\mathcal{P})$ heißt \emph{LL(1)-Grammatik}, genau dann wenn zu je zwei Produktionen $A \to \alpha$ und $A \to \beta$ gilt:
\begin{enumerate}
 \item $\operatorname{FIRST}(\alpha) \cap \operatorname{FIRST}(\beta) = \emptyset$
 \item Falls $\varepsilon \in \operatorname(\alpha)$ enthält, dann $\operatorname{FIRST}(\beta) \cap \operatorname{FOLLOW}(A) = \emptyset$
\end{enumerate}

\subsection{Tabellengesteuerter Kellerautomat}
Fehlersituation:
\begin{enumerate}
 \item Stackelement stimmt nicht mit gelesener Eingabe überein
 \item Eintrag in Parse-Tabelle ist leer
\end{enumerate}

Stelle zu gegebene Grammatik $G$ die Parsetabelle $\mathcal{M}$ wie folgt auf:
\begin{enumerate}
 \item Zu $A \to \alpha$ und $a \in \operatorname{FIRST}(\alpha), a \in \mathcal{T}$, trage $A \to \alpha$ in das Feld $\mathcal{M}[A,\alpha]$ ein.
 \item Wenn $\varepsilon \in \operatorname{FIRST}(\alpha)$ und $A \to \alpha \in \mathcal{P}$ mit $a \in \operatorname{FOLLOW}(A)$, trage $A \to \alpha$ in $\mathcal{M}[A,a]$ ein. Wenn $\varepsilon \in \operatorname{FIRST}(\alpha)$ und $\$ \in \operatorname{FOLLOW}(A)$, dann trage $A \to \alpha$ in $\mathcal{M}[A, \$]$ ein.
 \item Alle Felder, die leer bleiben werden als Fehlersituation interpretiert!
\end{enumerate}
\Bsp Parsetabellle zu Beispielgrammatik
\begin{center}
    \begin{tabular}{r||c|c|c|c|c|c|}
         & +                    & *            & (           & )                    & \textbf{id}         & \$                   \\\hline\hline
      E  &                      &              & $E \to TE'$ &                      & $E \to TE'$         &                      \\\hline
      E' & $E' \to +TE'$        &              &             &                      &                     &                      \\\hline
      T  &                      &              & $T \to FT'$ &                      & $E \to FT'$         &                      \\\hline
      T' & $T' \to \varepsilon$ & $T \to *FT'$ &             & $T' \to \varepsilon$ &                     & $T' \to \varepsilon$ \\\hline
      F  &                      &              & $F \to (E)$ &                      & $F \to \textbf{id}$ &                      \\\hline
    \end{tabular}
\end{center}
\Lemma $G$ ist eindeutig.
\Bew Parsetabelle zu $G$ enthält in jedem Feld höchstens \emph{einen} Eintrag.

 