\documentclass[a4paper,10pt]{scrartcl}
\usepackage[utf8x]{inputenc}
\usepackage[T1]{fontenc}
\usepackage{amsmath,amsfonts,amssymb,amscd,amsthm,xspace}
\usepackage[ngerman]{babel}
\usepackage{listingsutf8}
\usepackage{color}
\usepackage{geometry}
\usepackage{graphicx}
\usepackage{multicol}
\usepackage{pst-tree}

\geometry{a4paper, left=2cm,right=2cm,top=2cm,bottom=2cm}

\title{Übersetzerbau - Übungsblatt 4}
\author{Christian Cikryt (4285814; Tutorium: Mo. 16-18 Uhr)\\
  Robert Fels (4210496; Tutorium: Mi. 8-10 Uhr)\\
  Martin Lenders (4206090; Tutorium: Mi. 10-12 Uhr)
  }
\date{\today}

\newcommand{\changefont}[3]{\fontfamily{#1} \fontseries{#2} \fontshape{#3} \selectfont}

\renewcommand{\thesection}{Aufgabe \arabic{section}}
\renewcommand{\labelenumi}{\theenumi)}
\renewcommand{\theenumi}{\alph{enumi}}

\definecolor{lgray}{gray}{0.95}
\definecolor{purple}{rgb}{0.498,0,0.3333}
\definecolor{identifier}{rgb}{0,0,0.1}
\definecolor{string}{rgb}{0.192,0,1}
\definecolor{comment}{rgb}{0.25,0.5,0.37}

\pagestyle{myheadings}
\oddsidemargin\oddsidemargin
\markright{Martin Lenders, Christian Cikryt}

\lstset{
	tabsize=4, 
	frame=tlrb, 
	basicstyle=\footnotesize\changefont{pcr}{m}{n},
	breaklines=true,
	numbers=left,
	emphstyle=\textit, 
	language=Python,
	keywordstyle=\color{purple}\textbf, 
	identifierstyle=\color{identifier},
	stringstyle=\color{string},
	backgroundcolor=\color{lgray},
	showstringspaces=false,
	commentstyle=\color{comment},
	extendedchars=true,
	inputencoding=utf8/latin1
}
\psset{nodesep=2pt,levelsep=2em,treesep=2em}

\makeatletter
\newcommand{\Rm}[1]{\textrm{\expandafter\@slowromancap\romannumeral #1@}}
\makeatother
\newcommand{\print}[1]{\{\texttt{print(\dq #1\dq})\}}
\newcommand{\add}[1]{\{\texttt{s += #1})\}}

\begin{document}

\maketitle

\section{}
\psset{arrows=->,colsep=1cm,rowsep=0.5cm}
\begin{enumerate}
\item   NEA/DEA:
        \begin{center}
            \begin{psmatrix}[mnode=circle]
               [mnode=r]\  & [doubleline=true] $q_0$
            \end{psmatrix}
            \ncline{1,1}{1,2}\naput{start}
            \nccurve[angleA=-20,angleB=20,ncurv=4]{1,2}{1,2}\nbput{$a,b$}
        \end{center}
\item   NEA:
        \begin{center}
            \begin{psmatrix}[mnode=circle]
               [mnode=r]\  & $q_0$ & $q_1$ & $q_2$ & [doubleline=true] $q_3$
            \end{psmatrix}
            \ncline{1,1}{1,2}\naput{start}
            \nccurve[angleA=70,angleB=110,ncurv=4]{1,2}{1,2}\nbput{$a,b$}
            \nccurve[angleA=-20,angleB=20,ncurv=4]{1,5}{1,5}\nbput{$a,b$}
            \ncline{1,2}{1,3}\naput{$a$}
            \ncline{1,3}{1,4}\naput{$b$}
            \ncline{1,4}{1,5}\naput{$b$}
        \end{center}
        DEA:
        \begin{center}
            \begin{psmatrix}[mnode=circle]
               [mnode=r]\  & $q_0$ & $q_1$ & $q_2$ & [doubleline=true] $q_3$
            \end{psmatrix}
            \ncline{1,1}{1,2}\naput{start}
            \nccurve[angleA=70,angleB=110,ncurv=4]{1,2}{1,2}\nbput{$b$}
            \nccurve[angleA=70,angleB=110,ncurv=4]{1,3}{1,3}\nbput{$a$}
            \nccurve[angleA=-160,angleB=-20]{1,4}{1,3}\naput{$a$}
            \nccurve[angleA=-20,angleB=20,ncurv=4]{1,5}{1,5}\nbput{$a,b$}
            \ncline{1,2}{1,3}\naput{$a$}
            \ncline{1,3}{1,4}\naput{$b$}
            \ncline{1,4}{1,5}\naput{$b$}
        \end{center}
\end{enumerate}

\section{}
\begin{enumerate}
 \item  Linksableitung:
        \begin{align*}
         \underline S &  \to (\underline L) \to (\underline L,S) \to (\underline L,S,S) \to (\underline S,S,S) 
                         \to ((\underline L),S,S) \to ((\underline L,S),S,S) \to ((\underline S,S),S,S) \\
                      &  \to ((a,\underline S),S,S) \to ((a,a),\underline S,S) \to ((a,a),a,\underline S) 
                         \to ((a,a),a,(\underline L)) \to ((a,a),a,(\underline S)) \\
                      &  \to ((a,a),a,(a))
        \end{align*}
        Rechtsableitung:
        \begin{align*}
         \underline S &  \to (\underline L) \to (L,\underline S) \to (L,(\underline L)) \to (L, (\underline S)) 
                         \to (\underline L,(a)) \to (L,\underline S,(a)) \to (\underline L,a,(a)) \to ((\underline L),a,(a))\\
                      &  \to (((L,\underline S),a,(a)) \to ((\underline L,a),a,(a)) \to ((\underline S,a),a,(a)) \to ((a,a),a,(a))
        \end{align*}
\item   \hspace{0cm}\\
        \begin{center}
            \pstree{\Tr{$S$}}{
                \Tr{(}
                \pstree{\Tr{$L$}}{
                    \pstree{\Tr{$L$}}{
                        \pstree{\Tr{$S$}}{
                            \pstree{\Tr{$L$}}{
                                \Tr{(}
                                \pstree{\Tr{$S$}}{
                                    \pstree{\Tr{$L$}}{
                                        \pstree{\Tr{$S$}}{
                                            \Tr{$a$}
                                        }
                                    }
                                    \Tr{,}
                                    \pstree{\Tr{$S$}}{
                                        \Tr{$a$}
                                    }
                                }
                                \Tr{)}
                            }
                        }
                    }
                    \Tr{,}
                    \pstree{\Tr{$S$}}{
                        \Tr{$a$}
                    }
                }
                \Tr{)}
            }
        \end{center}
\item   eindeutig, expandiert nur in eine Rictung
\item   LISP-Listen oder Tupel (von Tupeln, von Tupeln, ...)
\end{enumerate}

\section{}
\begin{enumerate}
 \item $S \to 0S0\ |\ 1S1\ |\ \varepsilon\ |\ 0\ |\ 1$
 \item $S \to 0S1S\ |\ 1S0S\ |\ \varepsilon$
\end{enumerate}

\section{}
\begin{enumerate}
\item   \begin{align*}
            S &\to XS\ |\ \varepsilon & \Longleftrightarrow && S &\to \{X\} \\
            S &\to X\ |\ \varepsilon & \Longleftrightarrow && S &\to [X]
        \end{align*}
\item   \begin{align*}
            S &\to \underline{\textrm{if}}\ E\ \underline{\textrm{then}}\ S\ [\underline{\textrm{else}}\ S]\ |\ \underline{\textrm{begin}}\ L\ \underline{\textrm{end}} \\
            L &\to S\{; S\}
        \end{align*}
\end{enumerate}

\section{}
Gegenbeispiel für Eindeutigkeit
\begin{center}
 \underline{if} $E$ \underline{then} \underline{if} $E$ \underline{then} \underline{other} \underline{else} \underline{if} $E$  \underline{then} \underline{other} \underline{else} \underline{other}
\end{center}
\renewcommand{\labelenumi}{(\theenumi)}
\renewcommand{\theenumi}{\roman{enumi}}
\begin{enumerate}
 \item Syntaxbaum 1:\\
       \begin{center}
        \pstree{\Tr{$S$}}{
            \pstree{\Tr{$G$}}{
                \Tr{\underline{if}}
                \Tr{$E$}
                \Tr{\underline{then}}
                \pstree{\Tr{$G$}}{
                    \Tr{\underline{if}}
                    \Tr{$E$}
                    \Tr{\underline{then}}
                    \pstree{\Tr{$G$}}{
                        \Tr{\underline{other}}
                    }
                    \Tr{\underline{else}}
                    \pstree{\Tr{$S$}}{
                        \Tr{\underline{if}}
                        \Tr{$E$}
                        \Tr{\underline{then}}
                        \pstree{\Tr{$S$}}{
                            \pstree{\Tr{$G$}}{
                                \Tr{\underline{other}}
                            }
                        }
                    }
                }
                \Tr{\underline{else}}
                \pstree{\Tr{$S$}}{
                    \pstree{\Tr{$G$}}{
                        \Tr{\underline{other}}
                    }
                }
            }
        }
       \end{center}
\item   Syntaxbaum 2:
        \begin{center}
            \pstree{\Tr{$S$}}{
                \Tr{\underline{if}}
                \Tr{$E$}
                \Tr{\underline{then}}
                \pstree{\Tr{$S$}}{
                    \pstree{\Tr{$G$}}{
                        \Tr{\underline{if}}
                        \Tr{$E$}
                        \Tr{\underline{then}}
                        \pstree{\Tr{$G$}}{
                            \Tr{\underline{other}}
                        }
                        \Tr{\underline{else}}
                        \pstree{\Tr{$S$}}{
                            \Tr{\underline{if}}
                            \Tr{$E$}
                            \Tr{\underline{then}}
                            \pstree{\Tr{$G$}}{
                                \Tr{\underline{other}}
                            }
                            \Tr{\underline{else}}
                            \pstree{\Tr{$S$}}{
                                \pstree{\Tr{$G$}}{
                                    \Tr{\underline{other}}
                                }
                            }
                        }
                    }
                }
            }
        \end{center}
\end{enumerate}
\end{document}
